\message{ !name(paper.tex)}\documentclass[10pt,twocolumn,letterpaper]{article}

\usepackage{cvpr}
\usepackage{times}
\usepackage{epsfig}
\usepackage{graphicx}
\usepackage{amsmath}
\usepackage{amssymb}

% Include other packages here, before hyperref.

% If you comment hyperref and then uncomment it, you should delete
% egpaper.aux before re-running latex.  (Or just hit 'q' on the first latex
% run, let it finish, and you should be clear).
\usepackage[pagebackref=true,breaklinks=true,letterpaper=true,colorlinks,bookmarks=false]{hyperref}


% \cvprfinalcopy % *** Uncomment this line for the final submission

\def\cvprPaperID{****} % *** Enter the 3DIMPVT Paper ID here
\def\httilde{\mbox{\tt\raisebox{-.5ex}{\symbol{126}}}}

% Pages are numbered in submission mode, and unnumbered in camera-ready
\ifcvprfinal\pagestyle{empty}\fi
\begin{document}

\message{ !name(paper.tex) !offset(139) }
\subsection{Blind review}

Many authors misunderstand the concept of anonymizing for blind
review.  Blind review does not mean that one must remove citations to
one's own work---in fact it is often impossible to review a paper
unless the previous citations are known and available.

Blind review means that you do not use the words ``my'' or ``our''
when citing previous work.  That is all.  (But see below for
techreports)

Saying ``this builds on the work of Lucy Smith [1]'' does not say that
you are Lucy Smith, it says that you are building on her work.  If you
are Smith and Jones, do not say ``as we show in [7]'', say ``as Smith
and Jones show in [7]'' and at the end of the paper, include reference
7 as you would any other cited work.

An example of a bad paper just asking to be rejected:
\begin{quote}
  \begin{center}
    An analysis of the frobnicatable foo filter.
  \end{center}

  In this paper we present a performance analysis of our previous
  paper [1], and show it to be inferior to all previously known
  methods.  Why the previous paper was accepted without this analysis
  is beyond me.

  [1] Removed for blind review
\end{quote}


An example of an acceptable paper:

\begin{quote}
  \begin{center}
    An analysis of the frobnicatable foo filter.
  \end{center}

  In this paper we present a performance analysis of the paper of
  Smith \etal [1], and show it to be inferior to all previously known
  methods.  Why the previous paper was accepted without this analysis
  is beyond me.

  [1] Smith, L and Jones, C. ``The frobnicatable foo filter, a
  fundamental contribution to human knowledge''.  Nature 381(12),
  1-213.
\end{quote}

If you are making a submission to another conference at the same time,
which covers similar or overlapping material, you may need to refer to
that submission in order to explain the differences, just as you would
if you had previously published related work.  In such cases, include
the anonymized parallel submission~\cite{Authors12} as additional
material and cite it as
\begin{quote} [1] Authors. ``The frobnicatable foo filter'', F\&G 2012
  Submission ID 324, Supplied as additional material {\tt fg324.pdf}.
\end{quote}

Finally, you may feel you need to tell the reader that more details
can be found elsewhere, and refer them to a technical report.  For
conference submissions, the paper must stand on its own, and not {\em
  require} the reviewer to go to a techreport for further details.
Thus, you may say in the body of the paper ``further details may be
found in~\cite{Authors12b}''.  Then submit the techreport as
additional material.  Again, you may not assume the reviewers will
read this material.

Sometimes your paper is about a problem which you tested using a tool
which is widely known to be restricted to a single institution.  For
example, let's say it's 1969, you have solved a key problem on the
Apollo lander, and you believe that the CVPR70 audience would like to
hear about your solution.  The work is a development of your
celebrated 1968 paper entitled ``Zero-g frobnication: How being the
only people in the world with access to the Apollo lander source code
makes us a wow at parties'', by Zeus \etal.

You can handle this paper like any other.  Don't write ``We show how
to improve our previous work [Anonymous, 1968].  This time we tested
the algorithm on a lunar lander [name of lander removed for blind
review]''.  That would be silly, and would immediately identify the
authors. Instead write the following:
\begin{quotation}
  \noindent
  We describe a system for zero-g frobnication.  This system is new
  because it handles the following cases: A, B.  Previous systems
  [Zeus et al. 1968] didn't handle case B properly.  Ours handles it
  by including a foo term in the bar integral.

  ...

  The proposed system was integrated with the Apollo lunar lander, and
  went all the way to the moon, don't you know.  It displayed the
  following behaviours which show how well we solved cases A and B:
  ...
\end{quotation}
As you can see, the above text follows standard scientific convention,
reads better than the first version, and does not explicitly name you
as the authors.  A reviewer might think it likely that the new paper
was written by Zeus \etal, but cannot make any decision based on that
guess.  He or she would have to be sure that no other authors could
have been contracted to solve problem B.

FAQ: Are acknowledgements OK?  No.  Leave them for the final copy.


\begin{figure}[t]
  \begin{center}
    \fbox{\rule{0pt}{2in} \rule{0.9\linewidth}{0pt}}
    % \includegraphics[width=0.8\linewidth]{egfigure.eps}
  \end{center}
  \caption{Example of caption.  It is set in Roman so that mathematics
    (always set in Roman: $B \sin A = A \sin B$) may be included
    without an ugly clash.}
  \label{fig:long}
  \label{fig:onecol}
\end{figure}

\message{ !name(paper.tex) !offset(325) }

\end{document}
