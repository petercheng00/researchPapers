\documentclass[10pt,twocolumn,letterpaper]{article}

\usepackage{cvpr} \usepackage{times} \usepackage{epsfig}
\usepackage{graphicx} \usepackage{amsmath} \usepackage{amssymb}
\usepackage{subfig}

% Include other packages here, before hyperref.

% If you comment hyperref and then uncomment it, you should delete
% egpaper.aux before re-running latex.  (Or just hit 'q' on the first
% latex run, let it finish, and you should be clear).
\usepackage[pagebackref=true,breaklinks=true,letterpaper=true,colorlinks,bookmarks=false]{hyperref}


% \cvprfinalcopy % *** Uncomment this line for the final submission

\def\cvprPaperID{****} % *** Enter the 3DIMPVT Paper ID here
\def\httilde{\mbox{\tt\raisebox{-.5ex}{\symbol{126}}}}

% Pages are numbered in submission mode, and unnumbered in
% camera-ready
\ifcvprfinal\pagestyle{empty}\fi
\begin{document}

%%%%%%%%% TITLE
\title{Texture Mapping 3D Models of Indoor Environments with Noisy
  Camera Poses}

\author{First Author\\
  Institution1\\
  Institution1 address\\
  {\tt\small firstauthor@i1.org}
  % For a paper whose authors are all at the same institution, omit
  % the following lines up until the closing ``}''.  Additional
  % authors and addresses can be added with ``\and'', just like the
  % second author.  To save space, use either the email address or
  % home page, not both
  \and
  Second Author\\
  Institution2\\
  First line of institution2 address\\
  {\small\url{http://www.author.org/~second}} }

\maketitle
% \thispagestyle{empty}

%%%%%%%%% ABSTRACT
\begin{abstract}
  Automated 3D modeling of building interiors is useful in
  applications such as virtual reality and environment
  mapping. Applying textures to these models is an important step in
  generating photorealistic visualizations of data gathered by
  modeling systems.  The localization of cameras in such systems often
  suffer from inaccuracies, resulting in visible discontinuties when
  different images are projected adjacently onto a plane for
  texturing. We propose two approaches for reducing discontinuities
  during texture mapping, one to robustly accomodate images of all
  orientations, and one to take advantage of optimal situations where
  images have uniform orientation. The effectiveness of our approaches
  will be demonstrated on two indoor datasets.
\end{abstract}

%%%%%%%%% BODY TEXT
\section{Introduction}
\label{sec:introduction}
Three-dimensional modeling of indoor environments has a variety of
applications such as training and simulation for disaster management,
virtual heritage conservation, and mapping of hazardous sites. Manual
construction of these digital models can be time consuming, and as
such, automated 3D site modeling has garnered much interest in recent
years.

The aim of this paper is to present a solution for texture mapping the
3D models generated by indoor modeling systems, with specific
attention given to a human-operated system with high camera pose
errors and great variance in camera locations.

The first step in automated 3D modeling is the physical scanning of
the environment's geometry. An indoor modeling system must be able to
recover camera poses within an environment while simulatenously
reconstructing the 3D structure of the environment itself. This is
known as the simultaneous localization and mapping (SLAM) problem, and
is generally solved by taking readings from laser range scanners,
cameras, and inertial measurement units (IMUs) at multiple locations
within the environment.

Mounting such devices on a human-carried platform provides unique
advantages over vehicular-based systems in terms of agility and
portability. Unfortunately, human-operated systems also result in much
larger localization error. As a result, common methods for texture
mapping generally produce poor results, as later shown in Sections
\ref{sec:simpleTextureMapping} and
\ref{sec:existingApproaches}. Before discussing how to overcome these
challenges, we first provide an overview of the backpack modeling
system from which our test data has been obtained.

Our backpack-mounted modeling system contains five 2D laser range
scanners, two cameras, and an orientation sensor. The laser scanners
are mounted orthogonally and have a 30-meter range and a 270$^{\circ}$
field of view. The two cameras are equipped with fisheye lenses,
reaching an approximately 180$^{\circ}$ field of view, and are mounted
with one facing left and the other facing right. These cameras take
images at the rate of 5 Hz. The orientation sensor provides
orientation parameters at a rate of 180 Hz.

With the laser scanners active, the human operator wearing the
backpack takes great care to walk a path such that every wall in the
desired indoor environment is traversed and scanned lengthwise at
least once.

Using data gathered by the onboard sensors and multiple localization
and loop-closure algorithms, the backpack is then localized over its
data collection period, and a 3D point cloud of the surrounding
environment is constructed \cite{chen2010indoor, kua2012loopclosure,
  liu2010indoor}. Approximate normal vectors for each point in the
point cloud are then calculated by gathering neighboring points within
a small radius and processing them through principal component
analysis. These normal vectors allow for the classification and
grouping of adjacent points into structures such as walls, ceilings,
floors, and staircases. A RANSAC algorithm is then employed to fit
polygonal planes to these structured groupings of points, resulting in
a fully planar model \cite{sanchez2012point}. This model, consisting
of multiple 2D polygonal planes in 3D space, along with the set of
images captured by the backpack's cameras and their 3D poses, can be
considered the input to our texture mapping problem

The remainder of the paper is organized as follows. Section
\ref{sec:simpleTextureMapping} describes the general problem of
texture mapping and examines two simple mapping approaches and their
shortcomings.  Section \ref{sec:existingApproaches} demonstrates two
previous attempts at localization refinement, and demonstrates their
inadequacies for our datasets.  Section \ref{sec:proposedApproach}
presents our proposed approach to texture mapping, combining an
improved localization refinement process with two image selection
approaches. Section \ref{sec:resultsAndConclusions} contains results
and conclusions.



\section{Simple Texture Mapping}
\label{sec:simpleTextureMapping}
\begin{figure}
  \centering
  \includegraphics[height=2in]{Projection.pdf}
  \caption{Planes are specified in 3D space by four corners $C_1$ to
    $C_4$. Images are related to each plane through the camera
    matrices $P_{1..M}$. }
  \label{fig:projection}
\end{figure}

In all subsequent sections, we discuss the process of texture mapping
a single plane, as the texturing of each of our planes is independent
and can be completed in parallel.

The geometry of the texture mapping process for a plane is shown in
Figure \ref{fig:projection}.  As described earlier, we are provided
with a set of $M$ images with which we must texture our target
plane. Each image has a camera matrix $P_i$ for $i=1..M$, which
translates a 3D point in the world coordinate system to a 2D point or
pixel in image $i$'s coordinates. If the 3D world point is not
contained in the image, the 2D point will simply be outside of the
image boundaries. A camera matrix $P_i$ is composed of the camera's
intrinsic parameters, such as focal length and image center, as well
as extrinsic parameters which specify the rotation and translation of
the camera's position in 3D world coordinates at the time that image
$i$ is taken. These extrinsic parameters are determined by the
backpack hardware and localization algorithms mentioned earlier. A
point $X$ on the plane in 3D space can be related to its corresponding
pixel $x$ in image $i$ through the following equation:

\[
x=project(P_iX)
\]

where
\[X = \begin{pmatrix} x \\ y \\ z \end{pmatrix} \textrm{ and }
project(X) = \begin{pmatrix} x/z \\ y/z \end{pmatrix}
\]

For the sake of simplicity, we treat all planes as rectangles by
generating minimum bounding boxes for them. Since our final textures
are stored as standard rectangular images anyway, we can simply leave
the area between plane boundary and bounding box untextured, or crop
it out as needed. A plane to be textured is thus defined by a bounding
box with corners $C_i$ in world coordinates and a normal vector
indicating the front facing side of the plane. Our goal is to texture
this plane using images captured by the backpack system, while
eliminating any visual discontinuities or seams that would suggest
that the plane's texture is not composed of a single continuous image.

\subsection{Direct Mapping}
\label{sec:directMapping}

Ignoring the fact that the camera matrices $P_{1..M}$ are inaccurate,
we can texture the plane by discretizing it into small square tiles,
generally about 5 pixels across, and choosing an image to texture each
tile with. We choose to work with rectangular units to ensure that
borders between any two distinct images in our final texture are
either horizontal or vertical. Since most environmental features
inside buildings are horizontal or vertical, any seams in our texture
intersect them miniimally and are likely to be less noticeable.

In order to select an image for texturing tile $t$, we must first
gather a list of candidate images that contain all four of its
corners, which we can quickly check by projecting $t$ into each image
using the projection method above. Furthermore, each candidate image
must have been taken at a time when its camera had a clear
line-of-sight to $t$, which can be calculated using standard
ray-polygon intersection tests between the camera location, the center
of $t$, and other planes, all in world coordinates.

Once we have a list of candidate images for $t$, we must define a
scoring function in order to compare images and objectively select the
best one. Since camera localization errors compound over distance, we
wish to minimize the distance between cameras used for texturing and
our plane. Additionally, we desire images that are projected
perpendicularly onto the plane, maximizing the resolution and amount
of useful texture available in their projections.

These two criteria can be met by maximizing the function

\[
\frac{1}{d} \cdot (-1 \cdot C) \cdot N
\]

which minimizes camera angle $\alpha$ and distance $d$ as shown in
Figure \ref{fig:scoringFunction}.

\begin{figure}
  \centering
  \includegraphics[height=1.5in]{scoringFunction.pdf}
  \caption{Images are selected by minimizing camera angle $\alpha$ and
    distance $d$.}
  \label{fig:scoringFunction}
\end{figure}



\begin{figure}
  \centering
  \includegraphics[width=3in]{wall1_naive.jpg}
  \caption{The result of direct texture mapping based on locally
    optimized textures.}
  \label{fig:directMapping}
\end{figure}


As Figure \ref{fig:directMapping} demonstrates, this approach leads to
the best texture for each tile independently, but overall results in
many image boundaries with abrupt discontinuities, due to significant
misalignment between images.

\subsection{Mapping with Caching}
\label{sec:mappingWithCaching}
Since discontinuities occur where adjacent tiles select different
images that do not align, it makes sense to take into account image
selections made by neighboring tiles while selecting the best image
for a given tile. By using the same image across tile boundaries, we
eliminate the discontinuity altogether. If a tile is not visible in
images chosen by its neighbors, using similar images is likely to
result in less noticeable discontinuities.

Similar to a caching mechanism, we select the best image for a tile
$t$ by searching through two subsets of images for a good candidate,
before searching through the entire set. The first subset of images is
those selected by adjacent tiles that have already been textured. We
must first check which images can map to $t$, and then of those, we
make a choice according to the same scoring function in Figure
\ref{fig:scoringFunction}. Rather than blindly reusing this image, we
ensure it meets a threshold, which we set to $\alpha < 45^\circ$, to
be considered a good image.

If no good image is found, we then check our second subset of images,
which consists of images that were taken near the images in the first
subset, both spatially and temporally. These images are not the same
as the ones used for neighboring tiles, but they were taken at a
similar location and time, suggesting that their localization and
projection are very similar. Again, if no good image is found
according to the same threshold, we then must search the entire set of
candidate images.

\begin{figure}
  \centering
  \includegraphics[width=3in]{wall1_cache_full.jpg}
  \caption{The result of adding a caching system to locally optimized
    textures.}
  \label{fig:caching}
\end{figure}


The result of this caching approach is shown in figure
\ref{fig:caching}. As compared to \ref{fig:directMapping},
discontinuities have been reduced overall, but the amount of remaining
seams suggests that image selection alone cannot produce seamless
textures. Camera matrices, or the image projections themselves have to
be adjusted in order to reliably generate clean textures.

\section{Existing Approaches to Image-Aligned Texture Mapping}
\label{sec:existingApproaches}
In order to produce seamless texture mapping, either camera matrices
need to be refined such that their localization is pixel accurate,
resulting in a perfect mapping, or image stitching techniques need to
be applied to provide this illusion. Before examining these
approaches, we first obtain a set of images to work with.

Rather than perform camera or image adjustments across the many
thousands of images acquired in a typical data collection, we opt to
work with the more limited set of images corresponding to those chosen
by the direct mapping approach, without caching. This set of images
constitutes a good candidate set for generating a final seamless
texture since it meets three important criteria. First, it contains at least one image that covers each tile on our
plane; this ensures no holes in our final texture. Second, since
images are all selected according to the same scoring function in
Figure \ref{fig:scoringFunction}, they are taken at as much of a head-on angle as possible and should project
onto the plane in similar ways. Third, as a side result of the scoring
function, selected images are only good candidates for the tiles near
their center of projection. Thus, there should be plenty of overlap
between selected images, allowing for some degree of shifting without
resulting in holes, as well as area for blending between them. With
this set of images, we now review two existing approaches towards refining and
combining their projections.

\subsection{Image Mosaicing}
\label{sec:imageMosaicing}
\begin{figure}
  \centering
  \includegraphics[width=3in]{panoMy.jpg}
  \caption{Image mosaicing. }
  \label{fig:mosaic}
\end{figure}


When images of a plane are taken from arbitrary overlapping positions,
they are related by homography \cite{hz}. Thus, existing
homography-based image mosaicing algorithms are applicable
\cite{brown2007automatic}. However, errors can compound when long
chains of images are mosaiced together using these approaches. For
example, a pixel in the $n$th image in the chain must be translated
into the first image's coordinates by multiplying by the $3\times3$
matrix $H_1 H_2 H_3 ... H_n$. Any error in one of these homography
matrices is propagated to all further images until the chain is
broken. For some chains of images this can happen almost immediately
due to erroneous correspondence matches and the resulting image mosaic
is grossly misshapen.

Figure \ref{fig:mosaic} shows the output of the AutoStitch software
package which does homography-based image mosaicing. This plane is
nearly a best-case scenerio with many features spread uniformly across
it. Even so, the mosaicing produces errors that causes straight lines
to appear as waves on the plane. This image was generated after
careful hand tuning as well. Many planes that had fewer features
simply failed outright. Thus, image mosaicing is not a robust enough
solution for reliably texture mapping our dataset.

\subsection{Image-Based 3D Localization Refinement}
\label{sec:imageBased3DRefinement}
\begin{figure}
  \centering
  \includegraphics[width=3in]{Graph_crop.pdf}
  \caption{The graph-based localization refinement algorithm from [11]
    suffers from the problem of compounding errors. }
  \label{fig:graph}
\end{figure}

Another approach is to refine the camera matrices using image
correspondences to guide the process. Each image's camera matrix has 6
degrees of freedom that can be adjusted. Previous work on this problem
attempted to refine camera matrices by solving a non-linear
optimization problem \cite{liu2010indoor}. This process is specific to
the backpack system which generated our dataset, as it must be run
during backpack
localization\cite{liu2010indoor,chen2010indoor}. Unfortunately, this
approach suffers from a similar error propagation problem shown in
Figure \ref{fig:graph}.

\section{Proposed Method for Seamless Texture Mapping}
\label{sec:proposedApproach}
Our proposed approach towards texture mapping is a two-step
process. First, with the same input set of images as described in
Section \ref{sec:existingApproaches}, we perform image rotation and
shifting in order to maximize SIFT feature matches between images. We
then project these images onto our plane, applying textures either
with the tile caching method from Section
\ref{sec:mappingWithCaching}, or with the more specialized method
described in Section \ref{sec:seamMinimization}.

\subsection{Image Projection and Rotation}
\label{sec:projectionAndRotation}
Our localization approach begins with the projection of all our images
onto separate copies of our plane, such that no projected data is
covered up and lost. This is done in the same way as the approaches in
Section \ref{sec:simpleTextureMapping}. We then perform rotations on
these projections, as adjacent images with different orientations will
result in strong discontinuities.

These rotations are accomplished by using Hough transforms, which
detect the presence and orientation of linear features in our
images. Rather than match the orientation of such features in each
image, we simply apply rotations such that the strongest near-vertical
features are made completely vertical. This is effective for indoor
models, since strong features in indoor scenes usually consist of
parallel vertical lines corresponding to doors, wall panels,
rectangular frames, etc. If features in the environment are not
vertical, or are not parallel to eachother, this step is skipped.


\begin{figure}
  \centering
  \includegraphics[width=3in]{matches.jpg}
  \caption{SIFT feature matches between overlapping images.}
  \label{fig:matches}
\end{figure}



\subsection{Robust SIFT Feature Matching using RANSAC}
\label{sec:robustSIFTFeatureMatching}
Our next step is to fix misalignment between overlapping images. We do
this by first searching for corresponding points between all pairs of
overlapping images using SIFT feature matching
\cite{lowe1999object}. An illustration of this is given in Figure
\ref{fig:matches}. The SIFT matches allow us to determine $dx$ and
$dy$ distances between each pair of features for two images on the
plane, determining where they should be projected relative to
eachother.

Since indoor environments often contain repetitive features such as
floor tiles or doors, we need to ensure that our SIFT-based distances
are reliable. In order to mitigate the effect of incorrect matches and
outliers, the RANSAC framework \cite{fischler1981random} is used for a
robust estimate of the optimal $dx$ and $dy$ distances between two
images. The RANSAC framework handles the consensus-building machinery,
and requires a fitting function and a distance function. For this
application, the fitting function simply finds the average distance
between matches in a pair of images. Our distance function for a pair
of points is the difference between those points' SIFT match distance
and the average distance computed by the fitting function; we use a 10
pixel outlier threshold. This means that a SIFT match is labeled as an
outlier if its horizontal or vertical distance is not within 10 pixels
of the average distance computed by the fitting function.

\subsubsection{Refining Image Positions using Least Squares}
\label{sec:refiningImagePositions}
There are a total of $M^{2}$ possible pairs of images, though we only
generate distances between images that overlap at SIFT feature
points. Given these distances and the original image location
estimates, we can solve a least squares problem ($\textrm{min}_{\vec{\beta}}
||X \vec{\beta} - \vec{\gamma}||_2^2 $) to estimate the correct location of the images
on the plane. The $M$-dimensional vector $\vec{\beta}$ represents the unknown $x$
and $y$ locations of each image on the plane from $1 \dots M$. The
optimal $x$ and $y$ locations are obtained in the same way, so we
only consider the $x$ locations here:

% Draw least squares problem here.

\[\vec{\beta} =
\begin{pmatrix}
  x_1, & x_2, & x_3, & \cdots & x_{M-1}, & x_M
\end{pmatrix}
\]

The $N$ by $(M+1)$ dimensional matrix $X$ is constructed with one row for each pair of images
with measured distances produced by the SIFT matching stage. A row in
the matrix has a $-1$ and $1$ in the columns corresponding to the two
images in the pair. For example, the matrix below indicates that we
generated a SIFT-based distance between images 1 and 2, images 1 and
3, images 2 and 3, etc.

\[
X =
\begin{pmatrix}
  -1 & 1 & 0 & \cdots & 0 & 0\\
  -1 & 0 & 1 & \cdots & 0 & 0\\
  0 & -1 & 1 & \cdots & 0 & 0\\
  \vdots  & \vdots & \vdots & \ddots & \vdots  & \vdots\\
  0 & 0 & 0 & \cdots & 1 & 0 \\
  0 & 0 & 0 & \cdots & -1 & 1 \\
  1 & 0 & 0 & \cdots & 0 & 0 \\
\end{pmatrix}
\]

If only relative distances between images are included then there is
no way to determine the absolute location of any of the images and the
matrix becomes rank deficient. To fix this we choose the first image
to serve as the anchor for the rest, meaning all the absolute
distances are based on its original location. This is done by adding a
row with a $1$ in the first column and the rest zeros.

Finally, the $N$-dimensional observation vector $\vec{\gamma}$ is constructed using the
SIFT-based distances generated earlier in the matching stage. The
distances are denoted as $d_1 \dots d_N$ for $N$ SIFT-based
distances. The last element in the observation vector is the location
of the first image determined by its original noisy localization, from \cite{chen2010indoor, liu2010indoor}:

\[
\vec{y}^T =
\begin{pmatrix}
  d_{1,2}, &d_{1,3}, &d_{2,3}, &\hdots &d_{N-2,N-1}, &d_{N-1,N}, &x_1
\end{pmatrix}
\]

The $\vec{\beta}$ that minimizes $||X \vec{\beta} - \vec{\gamma}||_2^2$ results in a set of
image locations on the plane that best honors all the SIFT-based
distance measurements between images. In practice there are often
cases where there is a break in the chain of images, meaning that no
SIFT matches were found between one segment of the plane and
another. In this case we add rows to the $X$ matrix and observations
to the $\vec{\gamma}$ vector that contain the original noisy $x$ and $y$ distance
estimates generated by the localization algorithm \cite{chen2010indoor, liu2010indoor}. Another way to do
this is to add rows for all neighboring pairs of images and solve a
weighted least squares problem where the SIFT distances are given a
higher weight i.e. 1, and the noisy distances generated by the localization
algorithm \cite{chen2010indoor, liu2010indoor} are given a smaller weight i.e. 0.01.

After completing this same process for the $y$ dimension as well, and
making the resultant shifts, our images overlap and
match each other with far greater accuracy. Applying the tile caching method from Section \ref{sec:mappingWithCaching} on these re-localized images, results in the significant improvements shown in Figure \ref{fig:shifted}.

\begin{figure}
  \centering
  \includegraphics[width=3in]{wall1_cache_full_shifted.jpg}
  \caption{Localization refinement results in signifcantly fewer
    discontinuities in the final texture.}
  \label{fig:shifted}
\end{figure}



\subsection{Texture Mapping with Seam Minimization}
\label{sec:seamMinimization}

As mentioned earlier, our backpack system has 2 cameras with
$180^\circ$ fisheye lenses facing to the right and left. Since the
backpack operator only takes care to fully scan each wall and not
necessarily the entirety of each ceiling and floor, cameras at higher
angles with respect to plane normal vectors must be used for texturing
large areas of floor and ceiling planes. These high camera angles
translate into images that span extremely large areas once projected
onto their planes. Such image projections have good scores (from
Figure \ref{fig:scoringFunction}) for certain areas on the plane, but
much worse scores for other areas. This can be seen in Figure
\ref{fig:highCameraAngle}. The tiling approach used in Section
\ref{sec:simpleTextureMapping} was thus a good choice for texturing,
as it allowed us to only use the parts of image projections that were
good choices for their respective plane locations.

For wall planes however, we have images all taken from close distances
and more head-on angles, and thus much smaller near-rectangular
projections to work with. As a result, there is less deviation over
the score of each tile within an image, as well as over the average
scores of all images. This means that the scoring criteria from Figure
\ref{fig:scoringFunction} is less relevant to walls, which enjoy an
abundance of head-on images.

Thus, for planes with optimal images, rather than selecting the set of
best images, since all images are near in quality, we instead select
the best set of images, such that the selection together results in
the cleanest final texture. We will accomplish this by using entire
images where possible, and defining a cost function to minimize the
visibility of seams in our final texture.

\subsection{Occlusion Masking}
\label{sec:occlusionMasking}
To begin with, we need to ensure that our images contain only content
that should be mapped onto the target plane in question. The tiling
approach used previously only checks occlusion for each tile as it is
being textured. For our new approach, we use entire images, so we need
to perform occlusion checks over the entirety of each image to
determine available areas for texture mapping.

Fortunately, by virtue of our indoor environments, the vast majority
of surface geometry is either horizontal or vertical, with high
amounts of right angles. This means that after masking out occluded
areas, our image projections will remain largely rectangular. We can
thus be efficient by recursively splitting each image into rectangular
pieces, and performing the same occlusion checks used in the tiling
process where needed. To actually occlude out rectangles, we simply
remove their texture, as we will ensure that untextured areas are
never chosen for texture mapping.

\subsection{Image Selection}
\label{sec:imageSelection}
To determine the set of images that results in the cleanest texture,
we need a cost function to evaluate the visibiilty of seams between
images in our set. A straightforward cost function that accomplishes
this is the sum of squared pixel differences in overlapping regions
between all pairs of images after they have been aligned as described
in Section \ref{sec:projectionAndRotation}. Minimizing this cost
function encourages image boundaries to occur either in featureless
areas, such as bare walls, or in areas where images match extremely
well.

With the cost function defined, we must now select the set of images
for which the overall cost function is minimized. Since nearly all the
images for our optimal case are head on, the best strategy to minimize
seams is to choose as few images as possible while texturing a given
plane. Thus,to cover the entirety of a plane, our problem can be
defined as minimally covering a polygon i.e. the plane, using other
polygons of arbitrary geometry i.e. our image projections, with the
added constraint of minimizing our cost function between chosen
images.  This is a complex problem, though we can take a number of
steps to simplify it. Given that our wall-texture candidate images are
all taken from a head-on angle, and assuming only minor rotations are
made during localization refinement, we can reason that their
projections onto the plane are approximately rectangular, as in Figure
\ref{sec:goodCameraAngle}. By discarding the minor excess texture and
cropping them all to be rectangular, our problem becomes the
conceptually simpler one of filling a polygon with rectangles, such
that the sum of all edge costs between each pair of rectangles is
minimal. We thus also retain the advantages of working with
rectangular units, as explained in section
\ref{sec:simpleTextureMapping}.

The location and orientation of the cameras on our backpack is such
that our images nearly always contain the entirety of the floor to
ceiling range of wall planes. Images are therefore rarely projected
with one above the other when texturing wall planes, which correspond
to the optimal planes we are working with.  In essence, we need only
to ensure horizontal coverage of our planes, as our images provide
full vertical coverage themselves. We can thus construct a Directed
Acyclic Graph (DAG) from the images, with edge costs defined by our
cost function above, and solve a simple shortest path problem to find
an optimal subset of images with regard to the cost functions.

\begin{figure}
  \centering
  \includegraphics[width=3in]{dagCreation.pdf}
  \caption{DAG construction for the image selection process.}
  \label{fig:dagCreation}
\end{figure}

Figure \ref{fig:dagCreation} demonstrates the construction of a DAG
from overlapping images of a long hallway. Images are sorted by
horizontal location left to right, and become nodes in a
graph. Directed edges are placed in the graph from left to right
between images that overlap. The weights of these edges are determined
by the cost functions discussed previously. Next, we add two
artificial nodes, one start node representing the left border of the
plane, and one end node representing the right border of the
plane. The left(right) artificial node has directed edges with equal
cost $C_0$ to(from) all images that meet the left(right) border of the
plane.

We now solve the shortest path problem from the start node to
the end node \cite{dijkstra}. This provides a set of images
completely covering the plane horizontally, while minimizing the cost
of the seams between images.

In rare cases where the vertical dimension of the plane is not
entirely covered by a chosen image, we are left with a hole where no
image is chosen to texture.

SOMETHING HERE ABOUT CRAP CRAP

Rather than reverting to a 2D-coverage problem, we can elect to simply
fill the hole by selecting images to fill it in a greedy fashion with
respect to edge costs of the same cost function.

With this completed, we have now mapped every location on our plane to
at least one image, and have minimized the number of images, as well
as the discontinuity between their borders. In the next section, we
apply blending between images where they overlap, but for the sake of
comparison with the unblended tile caching method in Section
\ref{sec:mappingWithCaching}, we arbitrarily choose one image for
texturing where images overlap. Figure \ref{fig:compare_unblended}
compares the tile caching method against this seam minimization
method.

\begin{figure}
  \centering
  \includegraphics[width=3in]{wall1_cache_full_shifted.jpg}
  \includegraphics[width=3in]{wall1_dynprog_noblend.jpg}
  \caption{The tile caching approach is presented above the seam
    minimization approach.}
  \label{fig:compare_unblended}
\end{figure}


Though both methods provide quite accurate texturing thanks to the
alignment process, the seam minimization approach results in fewer
visible discontinuities, since it directly reduces the cost of each
image boundary, while the tile caching method uses a scoring function
that only approximates this effect. Furthermore, seam minimization
guarantees the best selection of images, while the sequential tile
caching method may select images early on that turn out to be poor
choices once subsequent tiles have been processed.

In the context of the backpack modeling system, we apply the seam
minimization approach on walls, due to its superior output when
provided with head-on images. Floors and ceilings however, given their
images taken at oblique angles, as shown in Figure \ref{fig:floor_suboptimal}, are textured using the tile caching method.

\subsection{Blending}
\label{sec:blending}
We now apply the same blending process on our two texturing methods:
localization refinement followed by either tile caching or seam
minimization.

Although our preprocessing steps and image selections in either method
attempt to minimize all mismatches between images, there are
unavoidable discontinuities due to different lighting conditions or
inaccuracies in planar geometry or projection. These can however be
treated and smoothed over by applying alpha blending over image seams.
Whether the units we are blending are rectangularly-cropped images or
rectangular tiles, we can apply the same blending procedure, as long
as we have a guaranteed overlap between units to blend over.

For the tile caching method, we can ensure overlap by texturing a
larger tile than needed for display. For example, for a rendered tile
$l_1 \times l_1$, we can associate it with a texture $(l_1 + l_2)
\times (l_1 + l_2)$ in size. For the seam minimization method, we have
already ensured overlap between images. To enforce consistent blending
however, it is beneficial to add a minimuum required overlap distance
while solving the shortest path problem in Section
\ref{sec:imageSelection}. If images overlap in a region greater than
the overlap distance, we only apply blending over an area equal to the
overlap distance.

We apply alpha blending to blend pixels
linearly across overlapping regions. Figure
\ref{fig:pipeline} shows our entire texture mapping pipeline and demonstrates the
final blended output of both approaches.

\begin{figure}
  \centering
  \includegraphics[width=3in]{pipeline.pdf}
  \caption{Our final texture processing pipeline, with the final
    output of both approaches.}
  \label{fig:pipeline}
\end{figure}


\section{Results and Conclusions}
\label{sec:resultsAndConclusions}
In this paper, we have developed an approach to texture map models with noisy camera localization data. We are
able to refine image locations based on feature matching, and robustly
handle outliers.  We generalized one approach to texture mapping to
any manner of planes and images, and successfully textured both simple
rectangular walls as well as complex floor and ceiling geometry. We
also presented an optimized texturing method that takes advantage of
our localization refinement process and produces more seamless textures on
planes where multiple head-on images are available. Each of these
approaches is highly modular, and easily tunable for different
environments and acquisition hardware.

Ceilings and floors textured with the tile caching approach, and walls
textured with the seam minimization approach, are displayed in Figure
\ref{fig:results}. A more detailed walkthrough demonstrating fully
textured 3D models using the approaches in this paper is available in
the accompanying video to this paper.

\begin{figure}
  \centering
  \includegraphics[width=3in]{4thfloor21.jpg}
  \includegraphics[width=3in]{4thfloor61.jpg}
  \includegraphics[width=2in]{4thfloor8.jpg}
  \includegraphics[width=3in]{fullmodel.png}
  \caption{Examples of our final texture mapping output for walls,
    floors, and ceilings}
  \label{fig:results}
\end{figure}

{\small \bibliographystyle{ieee} \bibliography{egbib} }


\end{document}
