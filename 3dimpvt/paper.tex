\documentclass[10pt,twocolumn,letterpaper]{article}

\usepackage{cvpr}
\usepackage{times}
\usepackage{epsfig}
\usepackage{graphicx}
\usepackage{amsmath}
\usepackage{amssymb}

% Include other packages here, before hyperref.

% If you comment hyperref and then uncomment it, you should delete
% egpaper.aux before re-running latex.  (Or just hit 'q' on the first latex
% run, let it finish, and you should be clear).
\usepackage[pagebackref=true,breaklinks=true,letterpaper=true,colorlinks,bookmarks=false]{hyperref}


% \cvprfinalcopy % *** Uncomment this line for the final submission

\def\cvprPaperID{****} % *** Enter the 3DIMPVT Paper ID here
\def\httilde{\mbox{\tt\raisebox{-.5ex}{\symbol{126}}}}

% Pages are numbered in submission mode, and unnumbered in camera-ready
\ifcvprfinal\pagestyle{empty}\fi
\begin{document}

%%%%%%%%% TITLE
\title{\LaTeX\ Author Guidelines for 3DIMPVT Proceedings}

\author{First Author\\
  Institution1\\
  Institution1 address\\
  {\tt\small firstauthor@i1.org}
  % For a paper whose authors are all at the same institution, omit
  % the following lines up until the closing ``}''.  Additional
  % authors and addresses can be added with ``\and'', just like the
  % second author.  To save space, use either the email address or
  % home page, not both
  \and
  Second Author\\
  Institution2\\
  First line of institution2 address\\
  {\small\url{http://www.author.org/~second}} }

\maketitle
% \thispagestyle{empty}

%%%%%%%%% ABSTRACT
\begin{abstract}
  Automated 3D modeling of building interiors is useful in
  applications such as virtual reality and environment
  mapping. Applying realistic textures to these models is an important
  step in generating accurate visualizations of data gathered by
  modeling systems.  Unfortunately, the localization of cameras in
  such systems often suffer from inaccuracies, resulting in visible
  discontinuties when different images are projected adjacently onto a
  plane for texturing. Previous approaches at minimizing these
  discontinuities do not robustly handle a wide range of camera
  locations and angles, and often suffer from error accumulation when
  stitching together multiple images. We propose two approaches for
  reducing discontinuities during texture mapping, one to robustly
  accomodate images of all orientations, and one to take advantage of
  optimal situations where images have uniform orientation.
\end{abstract}

%%%%%%%%% BODY TEXT
\section{Introduction}
Three-dimensional modeling of indoor environments has a variety of
applications such as training and simulation for disaster management,
virtual heritage conservation, and mapping of hazardous sites. Manual
construction of these digital models can be time consuming, and as
such, automated 3D site modeling has garnered much interest in recent
years.

The first step in automated 3d modeling is the physical scanning of
the environment's geometry. An indoor modeling system must be able to
calculate camera locations within an environment while simulatenously
reconstructing the 3D structure of the environment itself. This
problem is studied by the robotics and computer vision communities as
the simultaneous localization and mapping (SLAM) problem, and is
generally solved using a combination of laser range scanners, cameras,
and inertial measurement units (IMUs).

The aim of this paper is to present a solution for texture mapping the
3D models generated by indoor modeling systems, with specific
attention given to a human-operated system with higher localization
errors and greater variance in camera locations. The paper is
organized as follows. Section 2 provides an overview of the backpack
modeling system and data processing pipeline from which input data and
examples used throughout this paper originate. Section 3 describes the
general problem of texture mapping and examines two simple mapping
approaches and their problems.  Section 4 demonstrates previous
attempts at localization refinement, and presents our new approach to
image alignment. Section 5 combines our image alignment algorithm with
a simple image selection approach, as well as with a new method for
image selection that takes advantage of optimal images. Section 6
contains results.

\section{Backpack Modeling System}
Human-operated data acquisition systems provide unique advantages over
vehicular-mounted systems in terms of agility and
portability. Unfortunately, human-operated systems also suffer from a
lack of automation and stability, resulting in much higher data
variance and localization error. As a result, common methods for
texture mapping generally produce poor results, as shown in section
3. Before discussing how to overcome these challenges, we will first
provide an overview of the backpack modeling system from which our
test data was obtained.

\subsection{Data Acquisition Hardware}
The backpack modeling system that captured our data contains five 2D
laser range scanners, two cameras, an orientation sensor, and an
IMU. The laser scanners are mounted orthogonally and have a 30-meter
range and a 270$^{\circ}$ field of view. The two cameras are equipped
with fisheye lenses, reaching an approximately 180$^{\circ}$ field of
view, and are mounted with one facing left and one facing right. These
cameras take images at the rate of 5 Hz. The orientation sensor
provides orientation parameters at a rate of 180 Hz. The IMU provides
highly accurate measurements of all 6 DOF at 200 Hz, and is used as a
ground truth reference.

With the backpack actively scanning, the human operator wearing it
takes great care to walk a path such that every wall in the desired
indoor environment is passed and scanned lengthwise at least once.

\subsection{Environment Reconstruction}
Using data gathered by the onboard sensors and multiple localization
and loop-closure algorithms, the backpack is first localized over its
data collection period, and a 3D point cloud of the surrounding
environment is constructed based on the laser scanner readings
relative to the backpack \cite{chen2010indoor}. Approximate normal vectors for
each point in the point cloud are then calculated by gathering
neighboring points within a small radius and processing them through
principal component analysis. These normal vectors allow for the
classification and grouping of adjacent points into structures such as
walls, ceilings, floors, and staircases. A RANSAC algorithm is then
employed to fit polygonal planes to these structured groupings of
points, resulting in a fully planar model \cite{sanchez2012point}. This model,
consisting of multiple 2D polygonal planes in 3D space, along with the
set of images captured by the backpack's camera, can be considered the
input to our texture mapping problem.


\section{Simple Texture Mapping}

\begin{figure}
  \centering
  \includegraphics[height=2in]{Projection.pdf}
  \caption{Planes are specified in 3D space by four corners $C_1$ to
    $C_4$. Images are related to each plane through the camera
    matrices $P_{1..M}$. }
  % \caption{The plane is specified in 3D space by the four corners
  % $C_1$ to $C_4$. Images are related to the plane through the camera
  % matrices $P_{1..M}$. For example, the 3D point X in the world
  % coordinate system is related to the image point x shown in the
  % figure as $x = P_i X$.}
  \label{fig:projection}
\end{figure}

In all subsequent sections, we will discuss the process of texture
mapping a single plane, as the texturing of each of our planes is
completely independent and can be completed in parallel. 

The geometry of the texture mapping process for a plane is shown in
Figure 1.  As described in the previous section, we are provided with
a set of $M$ images with which we must texture our target plane. Each
image has a camera matrix $P_i$ for $i=1..M$, which translates a 3D
point in the world coordinate system to a 2D point or pixel in image
$i$'s coordinates. If the 3D world point is not contained in the image, the 2D
point will simply be outside of the image boundaries. A camera matrix
$P_i$ is composed of the camera's intrinsic parameters, such as focal
length and image center, as well as extrinsic parameters which specify
the rotation and translation of the camera's position in 3D world coordinates at the time that image $i$ was taken. These
extrinsic parameters are determined by the backpack hardware and
localization algorithms mentioned in Section 2. A point $X$ on the
plane in 3D space can be related to its corresponding pixel $x$ in
image $i$ through the following equation:

\[
x=project(P_iX)
\]

where
\[X = \begin{pmatrix} x \\ y \\ z \end{pmatrix} \textrm{ and }
project(X) = \begin{pmatrix} x/z \\ y/z \end{pmatrix}
\]

For the sake of simplicity, we treat all planes as rectangles by
generating minimum bounding boxes for them. Since our final textures will be
stored as standard rectangular images anyway, we can simply leave the
area between plane boundary and bounding box untextured, or crop it
out as needed. A plane to be textured is thus defined by a bounding
box with corners $C_i$ in world coordinates and a normal vector
indicating the front facing side of the plane. Our goal is to texture
this plane using images captured by the backpack, while eliminating
any visual discontinuities or seams that would suggest that the
plane's texture was not composed of a single continuous image.



\subsection{Direct Mapping}


Ignoring the fact that the camera matrices $P_{1..M}$ are inaccurate,
we can texture the plane by discretizing it into small square tiles,
generally about 5 pixels across, and picking an image to texture each
tile with. We choose to work with rectangular units to ensure that
borders between any two distinct images in our final texture will be
either horizontal or vertical. Since most environmental features
inside buildings are horizontal or vertical, any seams in our texture
will intersect them miniimally and be less noticeable.

In order to select an image for texturing tile $t$, we must first
gather a list of candidate images that contain all four of its
corners, which we can quickly check using the projection method
above. Furthermore, each candidate image must have been taken at a
time when its camera had a clear line-of-sight to the center of $t$,
which can be calculated using standard ray-polygon intersection tests
between the camera location, our section, and other planes, all in
world coordinates.

Once we have a list of candidate images for $t$, we must define a
scoring function in order to compare images and objectively select the
best one. Since camera localization errors compound over distance, we
wish to minimize the distance between cameras used for texturing and
our plane. Additionally, we desire images that project squarely onto
the plane, maximizing the resolution and amount of useful texture
available in their projections.

A scoring function to maximize these two criteria is shown in figure \ref{fig:scoringFunction}.

\begin{figure}
  \centering
  \includegraphics[height=1.5in]{scoringFunction.pdf}
  \caption{For each tile, we find the image that minimizes $d$ and $\alpha$ by maximizing the following equation: $(1/d \cdot (-1 \cdot C) \cdot N)$}
  \label{fig:scoringFunction}
\end{figure}



\begin{figure}
  \centering
  \includegraphics[width=3in]{wall1_naive.jpg}
  \caption{The result of direct texture mapping based on locally
    optimized textures.}
  \label{fig:directMapping}
\end{figure}


As Figure \ref{fig:directMapping} demonstrates, this approach leads to
the best texture for each tile independently, but overall results in
many image boundaries with abrupt discontinuities, due to significant
misalignment between images.

\subsection{Mapping with Caching}
Since discontinuities occur where adjacent tiles select different
images that do not match up well, it makes sense to take into account
image choices made by neighboring tiles while selecting the best image
for a tile. By using the same image across tile boundaries, we
eliminate the discontinuity altogether. If this is not possible, using
very similar images will result in less noticeable discontinuities.

Similar to a caching mechanism, we select the best image for a tile
$t$ by searching through two subsets of images for a good candidate
before searching through the entire set. The first subset of images is
the set of images selected by adjacent tiles that have already been
processed. We must first check which images can map to $t$, and then
of those, we make a choice according to the same scoring function in
figure \ref{fig:scoringFunction}. Rather than blindly reusing this
image, we ensure it meets a predefined score threshold to be
considered a good image. If no good image is found, we then check our
second subset of images, which consists of images that were taken near
the images in the first subset, both spatially and temporally. These
images are not the same as the ones used for neighboring tiles, but
they were taken at a similar location and time, suggesting that their
localization and projection will be very similar. Again, if no good
image is found according to a threshold, we then must search the
entire set of candidate images.

\begin{figure}
  \centering
  \includegraphics[width=3in]{wall1_cache_full.jpg}
  \caption{The result of adding a caching system to locally optimized textures.}
  \label{fig:caching}
\end{figure}


The results of this caching approach are shown above in figure
\ref{fig:caching}.  Discontinuities have been reduced overall, but the
amount of remaining seams suggests that image selection
alone cannot produce seamless textures. The image projections themselves will have to be adjusted in order to reliably produce clean textures.


\section{Localization Refinement}
In order to produce photorealistic texture mapping, either camera
matrices need to be refined such that their localization is pixel
accurate, resulting in a perfect mapping, or image stitching
techniques need to be applied to provide this illusion. Before
examining these approaches, we first obtain a set of images to work
with.

Rather than perform camera or image adjustments across many thousands
of images, we choose to work with the more limited set of images
corresponding to those chosen by the direct mapping approach, without
caching. This set of images constitutes a good candidate set for
generating a final seamless texture, since it meets three important
criteria. First, this set of images contains at least one image that
covers each tile on our plane. Thus, unless extreme changes in
localization occur, we can ensure that there will be no holes in our
final texture. Second, since images are all selected according to the
same scoring function in figure \ref{fig:scoringFunction}, we know
that our images are all taken at a head-on angle and should project
onto the plane in similar ways. Third, as a side result of the scoring
function, selected images are only good candidates for the tiles near
their center of projection. Thus, there should be plenty of overlap
between selected images, allowing for some degree of shifting as well
as blending between them. With this set of images, we now examine two
approaches towards refining and combining their projections, before demonstrating our own.

\subsection{Image Mosaicing}

\begin{figure}
  \centering
  \includegraphics[width=3in]{panoMy.jpg}
  \caption{Image mosaicing. }
  \label{fig:mosaic}
\end{figure}


When images of a plane are taken from arbitrary overlapping positions,
they are related by homography \cite{hz}. Thus, existing
homography-based image mosaicing algorithms are applicable
\cite{brown2007automatic}. However, errors can compound when long
chains of images are mosaiced together using these approaches. For
example, a pixel in the $n$th image in the chain must be translated
into the first image's coordinates by multiplying by the $3\times3$
matrix $H_1 H_2 H_3 ... H_n$. Any error in one of these homography
matrices is propagated to all further images until the chain is
broken. For some chains of images this can happen almost immediately
due to erroneous correspondence matches and the resulting image mosaic
is grossly misshapen.

Figure \ref{fig:mosaic} shows the output of the AutoStitch software
package which does homography-based image mosaicing. This plane is
nearly a best-case scenerio with many features spread uniformly across
it. Even so, the mosaicing produces errors that causes straight lines
to appear as waves on the plane. This image was generated after
careful hand tuning as well. Many planes that had fewer features simply
failed outright. This leads to the conclusion that image mosaicing is not a
robust enough solution for reliably texture mapping our dataset.

\subsection{Image-Based 3D Localization Refinement}

\begin{figure}
  \centering
  \includegraphics[width=3in]{Graph_crop.pdf}
  \caption{Using the graph-based localization refinement algorithm
    from [11] suffers from the problem of compounding errors. }
  \label{fig:graph}
\end{figure}

Another approach is to refine the camera matrices using image
correspondences to guide the process. Each image's camera matrix has 6
degrees of freedom that can be adjusted. Previous work on this problem attempted to refine camera
matrices by solving a non-linear optimization problem
\cite{liu2010indoor}. This process is specific to the backpack system
which generated our dataset, as it must be run during backpack
localization\cite{liu2010indoor,chen2010indoor}. Unfortunately, this
approach suffers from a similar error propagation problem shown in
Figure \ref{fig:graph}. In our new approach, we also refine the
placement of images using image correspondences. However, we do so in
two dimensions on the plane whereas this previous work did so over all
6 degrees of freedom. Refining in two dimensions on the plane is less
flexible in that it does not address projection errors, however, it
provides significant benefits while avoiding the error propagation
problem.


\subsection{Improved Localization through 2D Matching}
Our approach begins with the projection of all our images onto
separate copies of our plane, such that no projected data is covered
up and lost. This is done in the same way as the approaches in section
3. We then perform Hough transforms to rotate these projections
similarly, and then shift each image's projection such that its
overlapping regions with other projections are as seamless as
possible.

\subsection{Hough Transforms}
Our first step is to use Hough transforms to adjust the rotation of
projected images. Rather than try and match the orientation of
features in each image, we simply apply rotations such that any strong
near-vertical features are made completely vertical. This is effective
for indoor modeling, since strong features in indoor scenes usually
contain vertical lines corresponding to doors, wall panels,
rectangular frames, etc. If features in the environment are not
vertical, or there are few features overall, this step should be
skipped.


\subsection{SIFT Feature Matching}
Our next step is to try and fix misalignment between overlapping
images. We do this by first searching for corresponding points between
all pairs of overlapping images using SIFT feature matching
\cite{lowe1999object}. An illustration of this is given in Figure
\ref{fig:matches}. The SIFT matches allow us to determine
feature-based $x$ and $y$ distances between two images on the plane,
which will allow us to reevaluate where each image should be projected.

\subsubsection{Robust SIFT Distances using RANSAC}
Since indoor environments often contain repetitive features, such as floor tiles or doors, we need to ensure that our SIFT-based distances are
reliable. In order to diminish the effect of incorrect matches and outliers, the RANSAC
framework \cite{fischler1981random} is used for a robust estimate of
the $x$ and $y$ distances between the two images. The RANSAC framework
attempts to build a consensus among the SIFT matches between every pair of images, and tries to generate the horizontal and vertical distance between them while ignoring
the influence of outliers that would skew the results. The framework
handles the consensus-building machinery, and only requires that two
functions be specified: the fitting function and the distance
function. These functions are called for random subsets of the SIFT
matches until the best set of inliers is found. For this application,
the fitting function simply finds the average distance between
matches. If the matches are exactly correct and the image is frontal
and planar then the distances for various SIFT feature matches should
be the same. Our distance function for a pair of points is the
difference between those points' SIFT match distance and the average
distance computed by the fitting function. We specified a 10 pixel
outlier threshold to the framework. This means that a SIFT match is
labeled as an outlier if its horizontal or vertical distance is not within 10
pixels of the average distance computed by the fitting function.

\subsubsection{Refining Image Positions using Least Squares}

There are a total of $M^{2}$ possible pairs of images, though we only generate distances between images that overlap at SIFT feature
points. Given these distances and the original image location
estimates, we can solve a least squares problem ($\textrm{min}_{\beta}
||\beta X - y||_2^2 $) to estimate the correct location of the images
on the plane. The vector $\beta$ of unknowns represent the correct $x$ and
$y$ locations of each image on the plane from $1 \dots M$. The optimal $x$ and $y$ locations are calculated in the same way, so we
will only consider the $x$ locations here:

% Draw least squares problem here.

\[\beta =
\begin{pmatrix}
  x_1, & x_2, & x_3, & \cdots & x_{M-1}, & x_M
\end{pmatrix}
\]

The matrix $X$ is constructed with one row for each pair of images
with measured distances produced by the SIFT matching stage. A row in
the matrix has a $-1$ and $1$ in the columns corresponding to the two
images in the pair. For example, the matrix below indicates that we
generated a SIFT-based distance between images 1 and 2, images 1 and
3, images 2 and 3, etc.

\[
X =
\begin{pmatrix}
  -1 & 1 & 0 & \cdots & 0 & 0\\
  -1 & 0 & 1 & \cdots & 0 & 0\\
  0 & -1 & 1 & \cdots & 0 & 0\\
  \vdots  & \vdots & \vdots & \ddots & \vdots  & \vdots\\
  0 & 0 & 0 & \cdots & 1 & 0 \\
  0 & 0 & 0 & \cdots & -1 & 1 \\
  1 & 0 & 0 & \cdots & 0 & 0 \\
\end{pmatrix}
\]

If only relative distances between images are included then there is
no way to determine the absolute location of any of the images and the
matrix becomes rank deficient. To fix this we choose the first image
to serve as the anchor for the rest, meaning all the absolute
distances are based on its original location. This is done by adding a
row with a $1$ in the first column and the rest zeros.

Finally, the observation vector $y$ is constructed using the
SIFT-based distances generated earlier in the matching stage. The
distances are denoted as $d_1 \dots d_N$ for $N$ SIFT-based
distances. The last element in the observation vector is the location of the first image determined by its original localization.

\[
y^T =
\begin{pmatrix}
  d_{1,2}, &d_{1,3}, &d_{2,3}, &\hdots &d_{N-2,N-1}, &d_{N-1,N}, &x_1
\end{pmatrix}
\]

The $\beta$ that minimizes $||\beta X - y||_2^2$ results in a set of
image locations on the plane that best honors all the SIFT-based
distance measurements between images. In practice there are often cases where there is a
break in the chain of images, meaning that no SIFT matches were found
between one segment of the plane and another. In this case we add rows
to the $X$ matrix and observations to the $y$ vector that contain the
original $x$ and $y$ distance estimates generated by the localization
algorithm. Another way to do this is to add rows for all neighboring
pairs of images and solve a weighted least squares problem where the
SIFT distances are given a higher weight i.e. 1, and the distances
generated by the localization algorithm are given a smaller weight
i.e. 0.01.

After completing this same process for the $y$ dimension as well, and making the resultant shifts, our image projections should overlap and match eachother with far greater accuracy. 


\section{Improved Texture Mapping with Localization and Blending}
Now that our images have been relocalized for much greater accuracy relative to eachother, we can revisit the cached mapping approach from section 3.2. Since we now have higher confidence in the quality of image boundaries, we can also explore an alternate method that greatly reduces seams in cases where we have an abundance of optimal images.

\subsection{Localized Mapping with Caching}
In figure \ref{fig:shifted}, we see the same wall from section 3,
texture mapped using the caching method, both without and with
localization for comparison.

\begin{figure}
  \centering
  \includegraphics[width=3in]{wall1_cache_full_shifted.jpg}
  \caption{Localization refinement results in signifcantly fewer
    discontinuities in the final texture.}
  \label{fig:shifted}
\end{figure}


It should be evident that our localization adjustments resulted in a
major improvement. A great strength of our tiling approach for image
selection is that it runs regardless of plane geometry or orientation,
and can handle images of all types. For example, here is the approach
as run on a ceiling and floor plane, both of which contain a multitude
of non-optimal images due to the camera locations on the backpack.

% \begin{figure}
%   \centering
%   \includegraphics[width=3in]{localized_caching_ceiling.png}
%   \caption{Using the graph-based localization refinement algorithm
%   from [11] suffers from the problem of compounding errors. }
%   \label{fig:graph}
% \end{figure}

Given that the backpack system's hardware and the data collection
process do strive to collect optimal images for walls however, we can
be less reserved when texturing walls, and simply fall back to this
more robust approach as needed.

\subsection{Texture Mapping with Seam Minimization}
The cached tile approach selected images based on individual quality
and neighboring quality, in an effort to reduce seams between
tiles. In a more optimal case, as the backpack system provides for
walls, images all have near-rectangular projections and thus a great
score according to the scoring function in figure
\ref{fig:scoringFunction}. Thus, we can reason that rather than
selecting the set of best images, since all images are good, we should
instead select the best set of images, such that the selection results
in the cleanest texture possible. We will accomplish this by using
entire images where possible, defining a cost function between images,
and seeking to maximize our total cost.

\subsection{Occlusion Masking}
Before we proceed with using our images, we need to ensure that they
contain only content that should be mapped onto the target plane in
question. The tiling approach used previously only checked occlusion
for each tile as it was being textured, so we need to perform
occlusion checks over the entirety of each image, so we know which
areas are available for texture mapping. By performing the same
intersection tests used for tiles, we can determine whether individual
sections of an image are occluded.

Fortunately, by virtue of our indoor environments, the vast majority
of surface geometry is either horizontal or vertical, with high
amounts of right angles. This means that after masking out occluded
areas, our image projections will remain largely rectangular. We can
thus recursively split each image into rectangular pieces, and perform
occlusion checks, similar to the process done for tiles earlier. To
actually occlude out rectangles, we simply remove their texture, as we
will ensure that untextured areas are never chosen for texture
mapping.

\subsubsection{Inter-Image Cost Functions}
In order to objectively decide which set of images results in the
cleanest texture, we need a cost function such that we can evaluate
the visibiilty of seams between images in our set. A straightforward
cost function that accomplishes this is the sum of squared pixel
differences in overlapping regions between all pairs of
images. Minimizing this cost function encourages image boundaries to
occur in featureless areas, such as bare walls, as well as in areas
where images match extremely accurately.

Another possible cost function is overall edge energy, i.e. the sum of
the smoothed gradient over seams. Minimizing this encourages image
boundaries to be placed in featureless areas even more than the first
cost function. For the results shown throughout this paper, the first
cost function was used, as the regular occurrence of featureless areas
was not guaranteed in our datasets, and our image matches do in fact
match at the pixel level in many cases.

\subsubsection{Image Selection}

Now that we have a cost function defined, we mechanically select the
set of images for which the overall cost function is minimized. Since
we aim to cover the entirety of our plane, our problem is to minimally
cover a polygon(our plane), using other polygons of arbitrary geometry
(our image projections), with the added constraint of minimizing our
cost function between chosen images. This is a complex problem, though
we can take a number of steps to simplify it. Returning again to the
optimality of our situation when texturing walls however, we can make
a quick simplification for the sake of efficiency. Given that our
wall-texture candidate images were all taken from a head-on angle, and
assuming only minor rotations were made during localization
refinement, we can reason that their projections onto the plane are
approximately rectangular. By discarding the minor excess texture and
cropping them all to be rectangular, our problem becomes the
conceptually simpler problem of filling a polygon with rectangles,
such that the sum of all edge costs between each pair of rectangles is
minimal. We thus also retain axis-aligned image boundaries, along with
their advantages explained in section 3.

By taking one more shortcut we can simplify our problem down even
further. Since this approach is already assuming optimal images, we
know it will be applied only on walls, which are the focus of the
backpack modeling system. For wall planes, care is taken so that
images contain the entirety of their floor-to-ceiling range, and thus
we do not have wall images such that one should be projected
vertically above the other. In essence, we need only to ensure
horizontal coverage of our planes, as our images provide full vertical
coverage themselves. We can thus construct a Directed Acyclic Graph
(DAG) from the images, again with edge costs defined by our cost
function above, and solve a simple shortest path problem to find an
optimal subset of images with regard to the cost functions.

%\begin{figure}
%  \centering
%  \includegraphics[width=3in]{DynProg.pdf}
%  \caption{Image Selection is done by constructing a graph of sorted
%    images. Then we solve a shortest path problem where the edge
%    weights represent the cost of a seam between two overlapping
%    images.}
%  \label{fig:DynProg}
%\end{figure}

Figure \ref{fig:DynProg} demonstrates the construction of a DAG from
overlapping images of a long hallway. Images are sorted by horizontal
location left to right, and become nodes in a graph. Directed edges
are placed in the graph from left to right between images that
overlap. The weights of these edges are determined by the cost
functions discussed previously. Next, we add two artificial nodes, one
start node representing the left border of the plane, and one end node
representing the right border of the plane. The left artificial node
has directed edges with equal cost to all images that meet the left
border of the plane, and the right artificial node has directed edges
from all images that meet the right border of the plane.

We now solve the standard shortest path problem from the start node to
the end node. This provides a set of images that completely covers the
plane horizontally, while minimizing the cost of the seams between
images.

In rare cases where the vertical dimension of the plane is not
entirely covered by a chosen image, we are left with a hole where no
image was chosen to texture. Rather than reverting to a 2D-coverage
problem, we can elect to simply fill the hole by selecting images in a
greedy fashion with respect to the same cost function.

With this completed, we have now mapped every location on our plane to
at least one image, and have minimized the amount of images, as well
as the discontinuity between their borders. In the next section, we
will apply blending between images where they overlap, but for the
sake of comparison with the unblended tile caching method in section
5.1, we arbitrarily pick one image for texturing where images
overlap. The following images compare the tile caching method against
this seam minimization method.

%\begin{figure}
%  \centering
%  \includegraphics[height=2in]{compare_wall.png}
%  \caption{Planes are specified in 3D space by four corners $C_1$ to
%    $C_4$. Images are related to each plane through the camera
%    matrices $P_{1..M}$. }
  % \caption{The plane is specified in 3D space by the four corners
  % $C_1$ to $C_4$. Images are related to the plane through the camera
  % matrices $P_{1..M}$. For example, the 3D point X in the world
  % coordinate system is related to the image point x shown in the
  % figure as $x = P_i X$.}
%  \label{fig:compare_wall}
%\end{figure}


Though both methods provide quite accurate texturing thanks to the
localization refinement process, the seam minimization approach
results in fewer visible discontinuities, since it directly reduces
the cost of each image boundary, while the tile caching method uses a
scoring function that only approximates this effect. Furthermore, seam
minimization guarantees the best selection of images, while the tile
caching method may select images early on that turn out to be poor
choices once subsequent tiles have been processed. This can be a
problem when localization refinement does not work perfectly. Such an
effect can be seen in the following image.

In the context of the backpack modeling system, we apply the seam
minimization approach on walls, due to its superior output when
provided with optimal images. Floors and ceilings however, given their
suboptimal images which require segmentation, are textured using the
tile caching method.

\subsection{Blending}
Until now, blending of images has not been used, for the sake of clear
comparisons between texturing methods. We will now apply the same
blending process on our two final texturing methods: localization
refinement followed by either tile caching or seam minimization.

Although our preprocessing steps and image selections in either method
attempt to minimize all mismatches between images, there are cases,
where due to different lighting conditions or inaccuracies in geometry
or projection, where we simply have unavoidable discontinuities. These
can however be treated and smoothed over by applying alpha blending
over image seams.  Whether the units we are blending are
rectangularly-cropped images or rectangular tiles, we can apply the
same blending procedure, as long as we have a guaranteed overlap
between units.

For the tile caching method, we can ensure overlap by texturing a
larger tile than we really have. For example, for a tile $l_1 x l_1$
in size, we can make sure to associate it with a texture $(l_1 + l_2)
x (l_1 + l_2)$ in size. For the seam minimization method, we have
already ensured overlap between images. To enforce consistent blending
however, it is beneficial to add a required overlap distance while
solving the shortest path problem in section 5.2.3. If images do not
overlap in a region at least the length of this overlap distance, we
do not consider them to overlap at all. If images overlap in a region
greater than the overlap distance, we will only apply blending over an
area equal to the overlap distance.

Our alpha blending technique is widely used, and blends pixels
linearly across overlapping regions. For each pixel within such a
region, we simply scale its intensity (by a factor from 0 to 1),
based on its distance from the image's border, with a cap set at a fixed distance from the border. In this way, we interpolate between two
overlapping images, providing a gradual transition between
them. Increasing the blending distance results in a
smoother transition, but can also result in more blurriness. A blended and unblended version of each method is shown below for comparison.

%\begin{figure}
%5  \centering
 % \includegraphics[height=2in]{compare_blending.png}
%  \caption{Planes are specified in 3D space by four corners $C_1$ to
%    $C_4$. Images are related to each plane through the camera
%    matrices $P_{1..M}$. }
%  % \caption{The plane is specified in 3D space by the four corners
  % $C_1$ to $C_4$. Images are related to the plane through the camera
  % matrices $P_{1..M}$. For example, the 3D point X in the world
  % coordinate system is related to the image point x shown in the
  % figure as $x = P_i X$.}
%  \label{fig:compare_blending}
%\end{figure}

\section{Conclusion and Results}
In this paper, we have shown how to accurately texture map models,
even when provided with imprecise camera localization data. We have
generalized one approach to any manner of planes and images, and
successfully textured both simple rectangular walls as well as complex
floor and ceiling geometry. We have also presented an optimized
texturing method that takes advantage of our localization refinement
process and produces cleaner textures on planes where multiple head-on
images are available. Ceilings and floors textured with the robust
approach, and walls textured with the optimized approach, are
displayed in the remainder of this paper. A more detailed walkthrough
demonstrating fully textured models using the approaches in this paper
can be seen in the accompanying video to this paper.

{\small \bibliographystyle{ieee} \bibliography{egbib} }


\end{document}
