\message{ !name(paper.tex)}%  article.tex (Version 3.3, released 19 January 2008)
%  Article to demonstrate format for SPIE Proceedings
%  Special instructions are included in this file after the
%  symbol %>>>>
%  Numerous commands are commented out, but included to show how
%  to effect various options, e.g., to print page numbers, etc.
%  This LaTeX source file is composed for LaTeX2e.

%  The following commands have been added in the SPIE class 
%  file (spie.cls) and will not be understood in other classes:
%  \supit{}, \authorinfo{}, \skiplinehalf, \keywords{}
%  The bibliography style file is called spiebib.bst, 
%  which replaces the standard style unstr.bst.  

\documentclass[]{spie}  %>>> use for US letter paper
%%\documentclass[a4paper]{spie}  %>>> use this instead for A4 paper
%%\documentclass[nocompress]{spie}  %>>> to avoid compression of citations
%% \addtolength{\voffset}{9mm}   %>>> moves text field down
%% \renewcommand{\baselinestretch}{1.65}   %>>> 1.65 for double spacing, 1.25 for 1.5 spacing 
%  The following command loads a graphics package to include images 
%  in the document. It may be necessary to specify a DVI driver option,
%  e.g., [dvips], but that may be inappropriate for some LaTeX 
%  installations. 
\usepackage{graphicx}
\usepackage{subfig}
\usepackage{amsmath}
\usepackage{amssymb}
\usepackage{hyperref}

\title{Texture mapping 3D planar models of indoor environments with noisy camera poses} 

%>>>> The author is responsible for formatting the 
%  author list and their institutions.  Use  \skiplinehalf 
%  to separate author list from addresses and between each address.
%  The correspondence between each author and his/her address
%  can be indicated with a superscript in italics, 
%  which is easily obtained with \supit{}.

\author{Peter Cheng, Michael Anderson, Stewart He, Avideh Zakhor
\skiplinehalf
University of California, Berkeley\\
}

 

%%%%%%%%%%%%%%%%%%%%%%%%%%%%%%%%%%%%%%%%%%%%%%%%%%%%%%%%%%%%% 
%>>>> uncomment following for page numbers
% \pagestyle{plain}    
%>>>> uncomment following to start page numbering at 301 
%\setcounter{page}{301} 
 
\begin{document}

\message{ !name(paper.tex) !offset(357) }
\subsection{2D Feature Alignment}
\label{sec:robustSIFTFeatureMatching}
Our next step is to align overlapping images by searching for
corresponding points between all pairs of overlapping images. We use
SIFT features for their high detection rate, and choose to use feature
alignment rather than pixel or intensity-based alignment due to the
differences in lighting as well as possible occlusion among our
images, both of which feature alignment is less sensitive to
\cite{lowe1999object, mikolajczyk2005performance, szeliski2006image}.
We use SiftGPU \cite{siftgpu} for its high performance on both feature
detection as well as pairwise matching. These matches determine $d^x$
and $d^y$ distances between each pair of features for two image
projections, though these distances may not always be the same for
different features. Since indoor environments often contain repetitive
features such as floor tiles or doors, we need to ensure that
SIFT-based distances are reliable. First, we only perform alignment on
parts of images that overlap given the original noisy poses. Second,
we discard feature matches that correspond to an image distance
greater than 40 pixels from what the noisy poses estimate. In order to
utilize the remaining feature matches robustly, RANSAC
\cite{fischler1981random} is again used to estimate the optimal $d^x_{i,j}$
and $d^y_{i,j}$ distances between two images $i$ and $j$. For this
application, the RANSAC fitting function finds the average distance
between selected matches in a pair of images, and the distance
function for a pair of points is chosen to be the difference between
those points' SIFT match distance and the average distance computed by
the fitting function. We use a 10 pixel threshold, so that SIFT
matches are labeled as outliers if their horizontal or vertical
distances are not within 10 pixels of the average distance computed by
the fitting function.

We now use the RANSAC-calculated $d^x_{i,j}$ and $d^y_{i,j}$ distances
between each pair of images to refine their positions using weighted
linear least squares. There are a total of $M^{2}$ possible pairs of
images, though we only generate distances between images with SIFT
matches. Given these distances and the original image location
estimates, we can solve a least squares problem
($\textrm{min}_{\vec{\beta}} ||A \vec{\beta} - \vec{\gamma}||_2^2 $)
to estimate the location of the images on the plane. The
$M$-dimensional vector $\vec{\beta}$ represents the unknown $x$
location of each image on the plane for $1 \dots M$. The optimal $x$
and $y$ locations are obtained in the same way, so we only consider
the $x$ locations here:

\[\vec{\beta} =
\begin{pmatrix}
  x_1, & x_2, & x_3, & \cdots & x_{M-1}, & x_M
\end{pmatrix}
\]

The $N \times M$ matrix $A$ is constructed with one row for each pair
of images with measured distances produced by the SIFT matching
stage. A row in the matrix has a $-1$ and $1$ in the columns
corresponding to the two images in the pair. For example, the matrix
below indicates a SIFT-based distance between images 1 and 2, images 1
and 3, images 2 and 3, etc.
\[
A =
\begin{pmatrix}
  -1 & 1 & 0 & \cdots & 0 & 0\\
  -1 & 0 & 1 & \cdots & 0 & 0\\
  0 & -1 & 1 & \cdots & 0 & 0\\
  \vdots  & \vdots & \vdots & \ddots & \vdots  & \vdots\\
  0 & 0 & 0 & \cdots & 1 & 0 \\
  0 & 0 & 0 & \cdots & -1 & 1 \\
  1 & 0 & 0 & \cdots & 0 & 0\\
\end{pmatrix}
\]
If only relative distances between images are included, the absolute
location of the images can not be calculated, and the matrix is rank
deficient. From Section \ref{sec:geometryAlignment}, a number of
images are anchored to geometry points, in one or both dimensions, and
thus their locations can be used to fix the rest in place. In case no
anchor images exist, we simply arbitrarily pick an image, as we have
no other reference points to work with. In the above matrix, the first
image is set to be such an anchor, simply by placing a $1$ in its
column.

The $N$-dimensional observation vector $\vec{\gamma}$ is constructed
using the SIFT-based distances generated in the RANSAC matching
stage. Elements in the observation vector corresponding to anchor
images are simply their locations as determined by the original noisy
localization. Thus a $\vec{\gamma}$ corresponding to the above matrix
can be written as:

\[
\vec{\gamma}^T =
\begin{pmatrix}
  d_{1,2}, &d_{1,3}, &d_{2,3}, &\hdots &d_{N-2,N-1}, &d_{N-1,N}, &x_1
\end{pmatrix}
\]

The $\vec{\beta}$ that minimizes $||A \vec{\beta} -
\vec{\gamma}||_2^2$ results in a set of image locations on the plane
that best honors all the SIFT-based distance measurements between
images. This solution however does not make use of our noisy camera
poses, and will fail when no SIFT matches are found between one
segment of the plane and another. To account for this, we add rows to
the $A$ matrix and observations to the $\vec{\gamma}$ vector
corresponding to the original noisy distances. We then add weighting
to our least squares problem where the SIFT distances and anchor
values are given a high weight e.g. 1, while the noisy distances are
given a smaller weight e.g. 0.01.

After completing this same process for the $y$ dimension as well, and
making the resultant shifts, our images overlap and match each other
with far greater accuracy. Applying the simple mapping scheme in
Section \ref{sec:tileBasedMapping} to the same wall used in that
section results in Figure \ref{fig:compareAll}(b), which has far fewer
discontinuities, though errors due to lighting differences and
repeating features are still visible.

\message{ !name(paper.tex) !offset(650) }

\end{document} 
