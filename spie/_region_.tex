\message{ !name(paper.tex)}%  article.tex (Version 3.3, released 19 January 2008)
%  Article to demonstrate format for SPIE Proceedings
%  Special instructions are included in this file after the
%  symbol %>>>>
%  Numerous commands are commented out, but included to show how
%  to effect various options, e.g., to print page numbers, etc.
%  This LaTeX source file is composed for LaTeX2e.

%  The following commands have been added in the SPIE class 
%  file (spie.cls) and will not be understood in other classes:
%  \supit{}, \authorinfo{}, \skiplinehalf, \keywords{}
%  The bibliography style file is called spiebib.bst, 
%  which replaces the standard style unstr.bst.  

\documentclass[]{spie}  %>>> use for US letter paper
%%\documentclass[a4paper]{spie}  %>>> use this instead for A4 paper
%%\documentclass[nocompress]{spie}  %>>> to avoid compression of citations
%% \addtolength{\voffset}{9mm}   %>>> moves text field down
%% \renewcommand{\baselinestretch}{1.65}   %>>> 1.65 for double spacing, 1.25 for 1.5 spacing 
%  The following command loads a graphics package to include images 
%  in the document. It may be necessary to specify a DVI driver option,
%  e.g., [dvips], but that may be inappropriate for some LaTeX 
%  installations. 
\usepackage{graphicx}
\usepackage{subfig}
\usepackage{amsmath}
\usepackage{amssymb}
\usepackage{hyperref}

\title{Texture mapping 3D planar models of indoor environments with noisy camera poses} 

%>>>> The author is responsible for formatting the 
%  author list and their institutions.  Use  \skiplinehalf 
%  to separate author list from addresses and between each address.
%  The correspondence between each author and his/her address
%  can be indicated with a superscript in italics, 
%  which is easily obtained with \supit{}.

\author{Peter Cheng, Michael Anderson, Stewart He, Avideh Zakhor
\skiplinehalf
University of California, Berkeley\\
}

 

%%%%%%%%%%%%%%%%%%%%%%%%%%%%%%%%%%%%%%%%%%%%%%%%%%%%%%%%%%%%% 
%>>>> uncomment following for page numbers
% \pagestyle{plain}    
%>>>> uncomment following to start page numbering at 301 
%\setcounter{page}{301} 
 
\begin{document}

\message{ !name(paper.tex) !offset(374) }
\subsection{2D Feature Alignment}
\label{sec:robustSIFTFeatureMatching}
Our next step is to align overlapping images by searching for
corresponding points between all pairs of overlapping images. We use
SIFT features for their high detection rate, and choose to use feature
alignment rather than pixel or intensity-based alignment due to the
differences in lighting as well as possible occlusion among our
images, both of which feature alignment is less sensitive to
\cite{lowe1999object, mikolajczyk2005performance, szeliski2006image}.
We use SiftGPU \cite{siftgpu} for its high performance on both feature
detection as well as pairwise matching. These matches determine $d^x$
and $d^y$ distances between each pair of features for two image
projections, though these distances may not always be the same for
different features. Since indoor environments often contain repetitive
features such as floor tiles or doors, we need to ensure that
SIFT-based distances are reliable. First, we only perform alignment on
parts of images that overlap given the original noisy poses. Second,
we discard feature matches that correspond to an image distance
greater than 40 pixels from what the noisy poses estimate. In order to
utilize the remaining feature matches robustly, RANSAC
\cite{fischler1981random} is again used to estimate the optimal
$d^x_{i,j}$ and $d^y_{i,j}$ distances between two images $i$ and
$j$. We use a 10 pixel threshold for RANSAC, so that SIFT matches are
labeled as outliers if their distance is not within 10 pixels of the
sampled average distance.


We now use the $d^x_{i,j}$ and $d^y_{i,j}$ distances between each pair
of images to refine their positions using weighted linear least
squares. The variables we wish to solve for are the $x$ and $y$
positions of our images, while our equations are the SIFT-based
distances between pairs of images, images fixed to geometry with 0 or
1 degrees of freedom, and the original camera poses. Ignoring
weighting for now, an example setup for solving
$\textrm{min}_{\vec{\beta}} ||A \vec{\beta} - \vec{\gamma}||_2^2 $
with 3 images is shown below.


\[
A =
\begin{pmatrix}
  -1 & 1 & 0 & 0 & 0 & 0\\
  0 & 0 & 0 & -1 & 1 & 0\\
  0 & -1 & 1 & 0 & 0 & 0\\
  0 & 0 & 0 & 0 & -1 & 1\\
  1 & 0 & 0 & 0 & 0 & 0\\
  0 & 0 & 0 & 1 & 0 & 0\\
  0 & -m & 0 & 0 & 1 & 0\\
\end{pmatrix}
\]

\[\vec{\beta} =
\begin{pmatrix}
  x_1, & x_2, & x_3, & y_1, & y_2, & y_3
\end{pmatrix}
\]

\[
\vec{\gamma}^T =
\begin{pmatrix}
  dx_{1,2}, &dy_{1,2}, &dx_{2,3}, &dy_{2,3}, &t1_x, &t1_y, & -mx_2 +
  ty_2
\end{pmatrix}
\]


In this scenario, a SIFT-based distance of $dx_{1,2}$, $dy_{1,2}$ was
calculated between images 1 and 2. This corresponds to the first and
second of $A$, while the third and fourth row of $A$ represent the
same for images 2 and 3. Rows 5 and 6 of $A$ correspond to a fixed
location of $tx_1$, $ty_1$ for image 1, while Row 7 corresponds to a
constraint to a line of slope $m$, with current location $tx_2$,
$ty_2$, both results of the geometry alignment procedure in Section
\ref{sec:geometryAlignment}. If we do not have enough SIFT matches, or
lack images matched to geometry, our matrix becomes rank-deficient and
our problem cannot be solved. As a result, we add rows for each image
corresponding to the original noisy image locations, and downweight
them heavily, e.g. with a weight of 0.01 versus 1.

Because our problem is linear, it is quickly solved, and after
applying the resulting shifts, our images overlap and match each other
with far greater accuracy. Applying the simple mapping scheme in
Section \ref{sec:tileBasedMapping} to the same wall used in that
section results in Figure \ref{fig:compareAll}(b), which has far fewer
discontinuities, though errors due to lighting differences and
repeating features are still visible.

\message{ !name(paper.tex) !offset(667) }

\end{document} 
