\message{ !name(paper.tex)}%  article.tex (Version 3.3, released 19 January 2008)
%  Article to demonstrate format for SPIE Proceedings
%  Special instructions are included in this file after the
%  symbol %>>>>
%  Numerous commands are commented out, but included to show how
%  to effect various options, e.g., to print page numbers, etc.
%  This LaTeX source file is composed for LaTeX2e.

%  The following commands have been added in the SPIE class 
%  file (spie.cls) and will not be understood in other classes:
%  \supit{}, \authorinfo{}, \skiplinehalf, \keywords{}
%  The bibliography style file is called spiebib.bst, 
%  which replaces the standard style unstr.bst.  

\documentclass[]{spie}  %>>> use for US letter paper
%%\documentclass[a4paper]{spie}  %>>> use this instead for A4 paper
%%\documentclass[nocompress]{spie}  %>>> to avoid compression of citations
%% \addtolength{\voffset}{9mm}   %>>> moves text field down
%% \renewcommand{\baselinestretch}{1.65}   %>>> 1.65 for double spacing, 1.25 for 1.5 spacing 
%  The following command loads a graphics package to include images 
%  in the document. It may be necessary to specify a DVI driver option,
%  e.g., [dvips], but that may be inappropriate for some LaTeX 
%  installations. 
\usepackage{graphicx}
\usepackage{subfig}
\usepackage{amsmath}
\usepackage{amssymb}
\usepackage{hyperref}
\usepackage{float}
\title{Texture mapping 3D planar models of indoor environments with noisy camera poses} 

%>>>> The author is responsible for formatting the 
%  author list and their institutions.  Use  \skiplinehalf 
%  to separate author list from addresses and between each address.
%  The correspondence between each author and his/her address
%  can be indicated with a superscript in italics, 
%  which is easily obtained with \supit{}.

\author{Peter Cheng, Michael Anderson, Stewart He, and Avideh Zakhor
\skiplinehalf
University of California, Berkeley\\
}

 

%%%%%%%%%%%%%%%%%%%%%%%%%%%%%%%%%%%%%%%%%%%%%%%%%%%%%%%%%%%%% 
%>>>> uncomment following for page numbers
% \pagestyle{plain}    
%>>>> uncomment following to start page numbering at 301 
%\setcounter{page}{301} 
 
\begin{document}

\message{ !name(paper.tex) !offset(467) }
\subsection{2D Feature Alignment}
\label{sec:robustSIFTFeatureMatching}
The next step is to align the selected images from Section
\ref{sec:simpleTextureMapping} for each surface to each other by
searching for corresponding feature points between all pairs of
overlapping images. We use feature alignment rather than pixel or
intensity-based alignment due to the differences in lighting as well
as possible occlusion among images, both of which feature alignment is
less sensitive to \cite{lowe1999object, mikolajczyk2005performance,
  szeliski2006image}. We use SiftGPU \cite{siftgpu} for its high
performance on both feature detection as well as pairwise
matching. These matches determine $dx$ and $dy$ distances between each
pair of features for two image projections, though these distances may
not always be the same for different features. Since indoor
environments often contain repetitive features such as floor tiles or
doors, we need to ensure that SIFT-based distances are
reliable. First, we only align parts of images that overlap given the
original noisy poses. Second, we discard feature matches that
correspond to an image distance greater than 200 mm from what the
noisy poses estimate. In order to utilize the remaining feature
matches robustly, RANSAC \cite{fischler1981random} is again used to
estimate the optimal $dx_{i,j}$ and $dy_{i,j}$ distances between two
images $i$ and $j$. We use a 5 mm threshold for RANSAC, so that SIFT
matches are labeled as outliers if their distance is not within 5 mm
of the sampled average distance.


We now use the feature-based distances between each pair of images as
well as geometry alignment results from Section
\ref{sec:geometryAlignment} to refine all image positions using a
weighted linear least squares approach. An example setup for a
weighted linear least squares problem $\textrm{min}_{\vec{\beta}}
||W^\frac{1}{2}(A \vec{\beta} - \vec{\gamma})||_2^2 $ with 3 images is
as follows.

\[
A =
\begin{pmatrix}
  -1 & 1 & 0 & 0 & 0 & 0\\
  0 & 0 & 0 & -1 & 1 & 0\\
  0 & -1 & 1 & 0 & 0 & 0\\
  0 & 0 & 0 & 0 & -1 & 1\\
  0 & -m_2 & 0 & 0 & 1 & 0\\
  1 & 0 & 0 & 0 & 0 & 0\\
  0 & 0 & 0 & 1 & 0 & 0\\
  1 & 0 & 0 & 0 & 0 & 0\\
  0 & 0 & 0 & 1 & 0 & 0\\
  0 & 1 & 0 & 0 & 0 & 0\\
  0 & 0 & 0 & 0 & 1 & 0\\
  0 & 0 & 1 & 0 & 0 & 0\\
  0 & 0 & 0 & 0 & 0 & 1\\


\end{pmatrix}\quad
\vec{\beta} =
\begin{pmatrix}
  x_1, \\ x_2, \\ x_3, \\ y_1, \\ y_2, \\ y_3
\end{pmatrix}
\vec{\gamma} =
\begin{pmatrix}
  dx_{1,2}, \\ dy_{1,2}, \\ dx_{2,3}, \\ dy_{2,3}, \\ -m_2gx_2 + gy_2,
  \\ gx_1, \\ gy_1, \\ tx_1, \\ ty_1, \\ tx_2, \\ ty_2, \\ tx_3, \\
  ty_3
  
\end{pmatrix}
\vec{W} =
\begin{pmatrix}
  1, \\ 1, \\ 1, \\ 1, \\ 1, \\ 1, \\ 1, \\ 0.01, \\ 0.01, \\ 0.01, \\
  0.01, \\ 0.01, \\ 0.01
\end{pmatrix}
\]


The variables we wish to solve for are the $x_i$ and $y_i$ positions
of images, while equations are the feature-based distances between
pairs of images, images fixed to geometry with 0 or 1 degrees of
freedom, and the original noisy camera poses. In this scenario, a
feature-based distance of $dx_{1,2}$, $dy_{1,2}$ was calculated
between images 1 and 2. This corresponds to the first and second row
of $A$, while the third and fourth row of $A$ represent the same for
images 2 and 3. Rows 5 through 7 correspond to results of the geometry
alignment procedure in Section
\ref{sec:geometryAlignment}. Specifically, row 5 corresponds to a
geometry-based constraint of image 2's location to a line of slope
$m_2$, passing through point $gx_2$, $gy_2$, while rows 6 and 7
correspond to a fixed location for image 1 without any degrees of
freedom. Rows 8 through 13 correspond to the original camera poses for
each image ($tx_i,ty_i$).



The original camera poses are needed because we sometimes are not able
to detect feature matches in all of our images, or we lack enough
geometry alignment results to generate a single solution. Since we
want to minimally use our original poses, we give them a weighting
factor of $0.01$, while all other equations are weighted at $1$.


Since this problem is linear, it can be solved efficiently; after
applying the resulting shifts, images overlap and match each other
with far greater accuracy. Using the simple tile-based texturing
scheme from Section \ref{sec:simpleTextureMapping} on these adjusted
images results in Figure \ref{fig:compareAll}(b), which has far fewer
discontinuities than in \ref{fig:compareAll}(a), though some 3d error
as well as lighting differences and parallax effects are still
visible.

\message{ !name(paper.tex) !offset(767) }

\end{document} 
