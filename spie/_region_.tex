\message{ !name(paper.tex)}%  article.tex (Version 3.3, released 19 January 2008)
%  Article to demonstrate format for SPIE Proceedings
%  Special instructions are included in this file after the
%  symbol %>>>>
%  Numerous commands are commented out, but included to show how
%  to effect various options, e.g., to print page numbers, etc.
%  This LaTeX source file is composed for LaTeX2e.

%  The following commands have been added in the SPIE class 
%  file (spie.cls) and will not be understood in other classes:
%  \supit{}, \authorinfo{}, \skiplinehalf, \keywords{}
%  The bibliography style file is called spiebib.bst, 
%  which replaces the standard style unstr.bst.  

\documentclass[]{spie}  %>>> use for US letter paper
%%\documentclass[a4paper]{spie}  %>>> use this instead for A4 paper
%%\documentclass[nocompress]{spie}  %>>> to avoid compression of citations
%% \addtolength{\voffset}{9mm}   %>>> moves text field down
%% \renewcommand{\baselinestretch}{1.65}   %>>> 1.65 for double spacing, 1.25 for 1.5 spacing 
%  The following command loads a graphics package to include images 
%  in the document. It may be necessary to specify a DVI driver option,
%  e.g., [dvips], but that may be inappropriate for some LaTeX 
%  installations. 
\usepackage[]{graphicx}

\title{Texture mapping 3D planar models of indoor environments with noisy camera poses} 

%>>>> The author is responsible for formatting the 
%  author list and their institutions.  Use  \skiplinehalf 
%  to separate author list from addresses and between each address.
%  The correspondence between each author and his/her address
%  can be indicated with a superscript in italics, 
%  which is easily obtained with \supit{}.

\author{Peter Cheng, Michael Anderson, Stewart He, Avideh Zakhor
\skiplinehalf
University of California, Berkeley\\
}

 

%%%%%%%%%%%%%%%%%%%%%%%%%%%%%%%%%%%%%%%%%%%%%%%%%%%%%%%%%%%%% 
%>>>> uncomment following for page numbers
% \pagestyle{plain}    
%>>>> uncomment following to start page numbering at 301 
%\setcounter{page}{301} 
 
\begin{document}

\message{ !name(paper.tex) !offset(594) }
\subsection{Blending}
\label{sec:blending}
We now apply the same blending process to the two texturing methods:
tile caching or seam minimization.

Although the image alignment steps and image selection in both methods
attempt to minimize all mismatches between images, there are
occasional unavoidable discontinuities in the final texture due to
different lighting conditions or inaccuracies in planar geometry or
projection. These can however be treated and smoothed over by applying
alpha blending over image seams.  Whether the units we are blending
are rectangularly-cropped images or rectangular tiles, we can apply
the same blending procedure, as long as we have a guaranteed overlap
between units to blend over.

For the tile caching method, we can ensure overlap by texturing a
larger tile than needed for display. For example, for a rendered tile
$l_1 \times l_1$, we can associate it with a texture $(l_1 + l_2)
\times (l_1 + l_2)$ in size. We have found $l_2 = \frac{l_1}{2}$ to provide a
good amount of blending without blurring features. For the seam
minimization method, we have already ensured overlap between
images. To enforce consistent blending however, we add a minimum
required overlap distance while solving the shortest path problem in
Section \ref{sec:seamMinimization}. Additionally, if images overlap in
a region greater than the overlap distance, we only apply blending
over an area equal to the overlap distance.

After blending pixels linearly across overlapping regions using alpha
blending, texture mapping is complete. Figures \ref{fig:compareAll}(e)
and \ref{fig:compareAll}(f) show the blended versions of Figures
\ref{fig:compareAll}(c) and \ref{fig:compareAll}(d) respectively. It
is clear that the seam minimization approach still exhibits better
alignment and fewer seams than the tile-caching method, as Figure
\ref{fig:compareAll}(f) has the best visual quality among the textures
in Figure \ref{fig:compareAll}.

\message{ !name(paper.tex) !offset(604) }

\end{document} 
