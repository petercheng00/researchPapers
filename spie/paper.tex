%  article.tex (Version 3.3, released 19 January 2008)
%  Article to demonstrate format for SPIE Proceedings
%  Special instructions are included in this file after the
%  symbol %>>>>
%  Numerous commands are commented out, but included to show how
%  to effect various options, e.g., to print page numbers, etc.
%  This LaTeX source file is composed for LaTeX2e.

%  The following commands have been added in the SPIE class 
%  file (spie.cls) and will not be understood in other classes:
%  \supit{}, \authorinfo{}, \skiplinehalf, \keywords{}
%  The bibliography style file is called spiebib.bst, 
%  which replaces the standard style unstr.bst.  

\documentclass[]{spie}  %>>> use for US letter paper
%%\documentclass[a4paper]{spie}  %>>> use this instead for A4 paper
%%\documentclass[nocompress]{spie}  %>>> to avoid compression of citations
%% \addtolength{\voffset}{9mm}   %>>> moves text field down
%% \renewcommand{\baselinestretch}{1.65}   %>>> 1.65 for double spacing, 1.25 for 1.5 spacing 
%  The following command loads a graphics package to include images 
%  in the document. It may be necessary to specify a DVI driver option,
%  e.g., [dvips], but that may be inappropriate for some LaTeX 
%  installations. 
\usepackage[]{graphicx}

\title{Texture mapping 3D planar models of indoor environments with noisy camera poses} 

%>>>> The author is responsible for formatting the 
%  author list and their institutions.  Use  \skiplinehalf 
%  to separate author list from addresses and between each address.
%  The correspondence between each author and his/her address
%  can be indicated with a superscript in italics, 
%  which is easily obtained with \supit{}.

\author{Peter Cheng, Michael Anderson, Stewart He, Avideh Zakhor
\skiplinehalf
University of California, Berkeley\\
}

 

%%%%%%%%%%%%%%%%%%%%%%%%%%%%%%%%%%%%%%%%%%%%%%%%%%%%%%%%%%%%% 
%>>>> uncomment following for page numbers
% \pagestyle{plain}    
%>>>> uncomment following to start page numbering at 301 
%\setcounter{page}{301} 
 
\begin{document}
\maketitle

%%%%%%%%%%%%%%%%%%%%%%%%%%%%%%%%%%%%%%%%%%%%%%%%%%%%%%%%%%%%% 
\begin{abstract}
  Automated 3D modeling of building interiors is used in applications
  such as virtual reality and environment mapping. Incorporating
  images and texture mapping these models allows for photorealistic
  visualizations of the data collected by such modeling systems. Pose
  estimates obtained by these systems often suffer from inaccuracies,
  resulting in visible discontinuities when successive images are
  projected adjacently onto a surface for texturing. Existing methods
  to stitch images together are often computationally expensive and
  work independently of pose estimates and geometry information. We
  propose a method to refine camera poses using both existing
  estimates and geometry information, followed by two different
  methods to composite images together, based on the uniformity of
  available images. The effectiveness of our methods are demonstrated
  on a number of disparate indoor environments.
\end{abstract}

% >>>> Include a list of keywords after the abstract

\keywords{Texture Mapping, Reconstruction, Image Stitching, Mosaicing}

%%%%%%%%%%%%%%%%%%%%%%%%%%%%%%%%%%%%%%%%%%%%%%%%%%%%%%%%%%%%%
\section{INTRODUCTION}
\label{sec:introduction} % \label{} allows reference to this section
Three-dimensional modeling of indoor environments has a variety of
applications such as training and simulation for disaster management,
virtual heritage conservation, and mapping of hazardous sites. Manual
construction of these digital models can be time-consuming, and as a
result, automated 3D site modeling has garnered much interest in
recent years. CITATION

The first step in automated 3D modeling is the physical scanning of
the environment's geometry. An indoor modeling system must recover its
own poses within an environment while simultaneously reconstructing
the 3D structure of the environment itself. CITATION. This is known as
the simultaneous localization and mapping (SLAM) problem, and is
generally solved by taking readings from laser range scanners,
cameras, and inertial measurement units (IMUs) at multiple locations
within the environment.

In this paper, we work with data obtained from a backpack-mounted
system, carried by an ambulatory human. Such a system provides
advantages over more common wheel-mounted systems in terms of agility
and portability, but results in much higher localization error and
uncertainty overall. CITATION. As a result, common methods for texture
mapping generally produce poor results, leading to the development of
the approaches contained in this paper. Besides localization
information and images, environment geometry is required as well for
texture mapping. In this paper we work mainly with models obtained
through plane-fitting surfaces to point clouds generated by the
backpack system. CITATION. We have also developed an appropriate
method to work with near-planar models, as described in APPENDIX
SOMETHING.

Though our texture mapping procedure was designed with the
aforementioned system in mind, it is easily generalizable to any other
system, provided the required inputs of surface geometry, images with
rough extrinsic pose estimates, and either provided or estimated
intrinsic camera calibration matrices. An added benefit of our method
is its relatively fast running time and modularity of each step,
allowing for quick tuning of parameters and easy accomodation of new
input data given updates in localization or model-generation
algorithms.

The remainder of the paper is organized as follows. Section X explains
how input images are projected onto our geometry and simple approaches
towards texturing. Section X covers more sophisticated existing
approaches to image matching and stitching, and their performance on
our datasets. Section X contains our approach towards image matching,
followed by Section X, which describes compositing approaches. Section
X contains further results and conclusions.
%%%%%%%%%%%%%%%%%%%%%%%%%%%%%%%%%%%%%%%%%%%%%%%%%%%%%%%%%%%%%

\section{Simple Texture Mapping}
\label{sec:simpleTextureMapping}
\begin{figure}
  \centering
  \includegraphics[height=2in]{Projection.pdf}
  \caption{Planes are specified in 3D space by four corners $C_1$ to
    $C_4$. Images are related to each plane through the camera matrics
    $P_{1..m}$. }
  \label{fig:projection}
\end{figure}

The geometry of the texture mapping process for a plane is shown in
Figure \ref{fig:projection}.  As described earlier, we are provided
with a set of $M$ images to texture the target plane. Each image has a
camera matrix $P_i$ for $i=1..M$, which translates a 3D point in the
world coordinate system to a 2D point or pixel in image $i$'s
coordinates. A camera matrix $P_i$ is composed of the camera's
intrinsic parameters, such as focal length and image center, as well
as extrinsic parameters which specify the rotation and translation of
the camera's position in 3D world coordinates at the time that image
$i$ is taken. These extrinsic parameters are determined by the
backpack hardware and localization algorithms \cite{chen2010indoor,
  liu2010indoor, kua2012loopclosure} and are quite noisy.

Because our backpack system takes photos at a rate of 5 Hz, thousands
of images are available for texturing each surface in our model. Our
goal in designing a texture mapping process is to decide which of
these images should be used, and where their contents should map onto
the texture, in order to minimize any visual discontinuities or seams
that would suggest that the plane's final texture is not composed of a
single continuous image.

\subsection{Direct Mapping}
\label{sec:directMapping}

Ignoring the fact that the camera matrices $P_{1..M}$ are inaccurate,
we can texture the plane by discretizing it into small square tiles,
in our case 5 pixels across, and choosing an image to texture each
tile.

We choose to work with rectangular units to ensure that borders
between any two distinct images in the final texture are either
horizontal or vertical. Since most strong environmental features
inside buildings are horizontal or vertical, any visible seams in our
texture intersect them minimally and are less noticeable.

In order to select an image for texturing a tile $t$, we must first
gather a list of candidate images that contain all four of its
corners, which we can quickly check by projecting $t$ into each image
using the $P_i$ camera matrices. Furthermore, each candidate image
must have been taken at a time when its camera had a clear
line-of-sight to $t$, which can be calculated using standard
ray-polygon intersection tests between the camera location, $t$, and
every other plane \cite{rayintersection}.

Once we have a list of candidate images for $t$, we define a scoring
function in order to objectively select the best image for texturing
$t$. Since camera pose errors compound over distance, we wish to
minimize the distance between cameras and the surfaces they
texture. Additionally, we desire images that are projected
perpendicularly onto the plane, maximizing the resolution and amount
of useful texture available in their projections, as well as
minimizing any parallax effects due to real-world geometry not
represented by our digital model. In other words, we wish to minimize
the angle between the tile's normal vector and the camera axis for
images selected for texture mapping. These two criteria can be met by
maximizing the function $\frac{1}{d} (-1 \cdot \vec{c}) \cdot \vec{n}$
as shown in Figure \ref{fig:scoringFunction}. Specifically, $d$ is the
distance between the centers of a camera and a tile, and $\vec{n}$ and
$\vec{c}$ are the directions of the plane's normal and the camera axis
respectively.

\begin{figure}
  \centering
  \includegraphics[height=1.5in]{scoringFunction.jpg}
  \caption{We minimize camera angle $\alpha$ and distance $d$ by
    maximizing the scoring function $\frac{1}{d} (-1 \cdot \vec{c})
    \cdot \vec{n}$}
  \label{fig:scoringFunction}
\end{figure}



\begin{figure}[h!]
  \centering \subfloat[][]{\includegraphics[width=3in,
    height=0.97in]{wall2_naive.pdf}}

  \centering \subfloat[][]{\includegraphics[width=3in,
    height=0.97in]{wall2_cache.pdf}}

  \centering \subfloat[][]{\includegraphics[width=3in,
    height=0.97in]{wall2_cache_shift.pdf}}

  \centering \subfloat[][]{\includegraphics[width=3in,
    height=0.97in]{wall2_dynprog_noblend.pdf}}

  \centering \subfloat[][]{\includegraphics[width=3in,
    height=0.97in]{wall2_cache_shift_blend.pdf}}

  \centering \subfloat[][]{\includegraphics[width=3in,
    height=0.97in]{wall2_dynprog.pdf}}
  \caption{(a) Direct mapping. (b) Mapping with caching. (c) Mapping
    with caching after image alignment. (d) Seam minimization after
    image alignment). (e) same as (c) with blending. (f) same as (d)
    with blending.}
  \label{fig:compareAll}
\end{figure}


As Figure \ref{fig:compareAll}(a) demonstrates, this approach leads to
the best texture for each tile independently, but overall results in
many image boundaries with abrupt discontinuities, due to significant
misalignment between images, as a result of camera pose inaccuracies.

\subsection{Mapping with Caching}
\label{sec:mappingWithCaching}
Since discontinuities occur where adjacent tiles select non-aligned
images, it makes sense to take into account image selections made by
neighboring tiles while texture mapping a given tile. By using the
same image across tile boundaries, we can eliminate a discontinuity
altogether. If this is not possible because a tile is not visible in
images used by neighboring tiles, using similar images across tile
boundaries also leads to less noticeable discontinuities.

Essentially a caching mechanism, we select the best image for a tile
$t$ by searching through two subsets of images for a viable candidate,
before searching through the entire set of valid images. The first
subset of images is the images selected by adjacent tiles that have
already been textured. We must first check which of these images can
map to $t$, and then of those, we make a choice according to the
scoring function in Figure \ref{fig:scoringFunction}. Before reusing
this image, we ensure it meets the criteria $\alpha < 45^\circ$, in
order to be considered a viable image, with $\alpha$ as the camera
angle as shown in Figure \ref{fig:scoringFunction}.

If no satisfactory image is found in the first subset, we check the
second subset of images, consisting of those taken near the ones in
the first subset, both spatially and temporally. These images are not
the same as the ones used for neighboring tiles, but are taken at a
similar location and time, suggesting that their localization and
projection are quite similar, allowing us to minimize any differences
in localization error as well as a parallax effects. Again, if no
viable image is found according to the same criteria, we search the
entire set of candidate images, selecting based on the same scoring
function from Figure \ref{fig:scoringFunction}.

The result of this caching approach is shown in Figure
\ref{fig:compareAll}(b). As compared to Figure
\ref{fig:compareAll}(a), discontinuities are reduced overall, but the
amount of remaining seams suggests that despite our high volume of
images, image selection alone cannot produce seamless textures. Camera
matrices, or the image projections themselves, have to be adjusted in
order to reliably generate clean textures.


\section{Existing Approaches to Image Alignment}
\label{sec:existingApproaches}
Stitching together multiple images to produce a larger image is a
commonly performed task, with many successful approaches over the past
few decades. Parts of images are first matched to eachother, through
direct pixel comparisons, or more commonly through feature detection
and matching. Images are then adjusted to maximize matches, either by
directly calculating homographies between pairs of images, or by
modifying camera poses from 1 to 6 degrees of freedom.

Feature detection and matching works best when multiple unique visual
references exist in the environment and are present within multiple
images. This allows for accurate matching when texturing detailed
objects at high resolutions CiTATION, or when creating outdoor
panoramas CITATION. In contrast, our indoor environments have a high
prevalence of bare surfaces, as well as repeating textures that cause
difficulty in distinguishing features. The lack of strong reference
points results in high uncertainty when matching images together.

Additionally, our datasets often contain long chains of images, which
leads to error accumulation when image correspondences are not
accurate. For example, when matching a long chain of images through
homographies, a pixel in the $nth$ image must be translated into the
first image's coordinates by multiplying by the $3\times3$ matrix $H_1
H_2 H_3 ... H_n$. Any error in one of these homography matrices is
propagated to all further images, resulting in drift. Figure
\ref{fig:mosaic3D}(b) shows the output of the AutoStitch software
package, which performs homography-based image mosaicing
\cite{autostitch}. Even with many features spread across this surface,
the mosaicing produces errors that cause straight lines to appear as
distorted, despite the fact that it was generated after careful hand
tuning. Many areas with fewer visual features simply failed outright
using this approach.

\begin{figure}
  \centering
  \subfloat[][]{\includegraphics[width=3in]{Graph_crop.pdf}}

  \centering \subfloat[][]{\includegraphics[width=3in]{panoMy.jpg}}
  \caption{Texture alignment via (a) the graph-based localization
    refinement algorithm from \cite{chen2010indoor} and (b) image
    mosaicing.}
  \label{fig:mosaic3D}
\end{figure}

Image projections can also be aligned by iteratively adjusting camera poses to maximize matches. This is generally done over 6 degrees of freedom. CITATION. Applying the approach in the LIU2010INDOOR paper however, results again in error accumulation and drift as shown in Figure \ref{fig:mosaic3D}(a). Furthermore, such nonlinear optimization approaches have an extremely high complexity and runtime, and the entire solution must be recomputed given any changes in input data.







\section{Our Image Alignment with Cool Name}
\section{Image Compositing}
\section{Results and Conclusions}


%%%%%%%%%%%%%%%%%%%%%%%%%%%%%%%%%%%%%%%%%%%%%%%%%%%%%%%%%%%%%
%poop below
\section{PARTS OF MANUSCRIPT}

This section describes the normal structure of a manuscript and how
each part should be handled.  The appropriate vertical spacing between
various parts of this document is achieved in LaTeX through the proper
use of defined constructs, such as \verb|\section{}|.  In LaTeX,
paragraphs are separated by blank lines in the source file.

At times it may be desired, for formatting reasons, to break a line
without starting a new paragraph.  This situation may occur, for
example, when formatting the article title, author information, or
section headings.  Line breaks are inserted in LaTeX by entering
\verb|\\| or \verb|\linebreak| in the LaTeX source file at the desired
location.

%%%%% Sometimes it is necessary to precede the double slash by
%%%%% \verb|\protect| to get the desired result, for example, in
%%%%% article titles.

%% -----------------------------------------------------------
\subsection{Title and Author Information}
\label{sec:title}

The article title appears centered at the top of the first page.  The
title font is 16 point, bold.  The rules for capitalizing the title
are the same as for sentences; only the first word, proper nouns, and
acronyms should be capitalized.  Avoid using acronyms in the title.
Keep in mind that people outside your area of expertise might read
your article.  Appendix~\ref{sec:misc} contains more about acronyms.

The list of authors immediately follows the title after a blank
vertical space of about 7 mm.  The font is 12 point, normal with each
line centered.  The authors' affiliations and addresses follow the
author list after another blank space of about 4 mm, also in 12-point,
normal font and centered.  Do not use acronyms in affiliations and
addresses. For multiple affiliations, each affiliation should appear
on a new line, separated from the following address by a semicolon.
Italicized superscripts may be used to identify the correspondence
between the authors and their respective affiliations.  Further author
information, such as e-mail address, complete postal address, and
web-site location, may be provided in a footnote by using
\verb|\authorinfo{}|, as demonstrated above.

When the abbreviated title or author information is too long to fit on
one line, multiple lines may be used; insert line breaks appropriately
to achieve a visually pleasing format.  The proper spacing of one and
one-half lines between the title, author list, and their affiliations
is achieved with the command \verb|\skiplinehalf| defined in {\tt
  spie.cls}.

%% -----------------------------------------------------------
\subsection{Abstract and Keywords}
The title and author information is immediately followed by the
Abstract. The Abstract should concisely summarize the key findings of
the paper.  It should consist of a single paragraph containing no more
than 200 words.  The Abstract does not have a section number.  A list
of up to ten keywords should immediately follow the Abstract after a
blank line.  These keywords will be included in a searchable database
at SPIE.

%% -----------------------------------------------------------
\subsection{Body of Paper}
The body of the paper consists of numbered sections that present the
main findings.  These sections should be organized to best present the
material.  See Sec.~\ref{sec:sections} for formatting instructions.

%% -----------------------------------------------------------
\subsection{Appendices}
Auxiliary material that is best left out of the main body of the
paper, for example, derivations of equations, proofs of theorems, and
details of algorithms, may be included in appendices.  Appendices are
enumerated with uppercase Latin letters in alphabetic order, and
appear just before the Acknowledgments and References.

%% -----------------------------------------------------------
\subsection{Acknowledgments}
In the Acknowledgments section, appearing just before the References,
the authors may credit others for their guidance or help.  Also,
funding sources may be stated.  The Acknowledgments section does not
have a section number.

%% -----------------------------------------------------------
\subsection{References}
The References section lists books, articles, and reports that are
cited in the paper.  It does not have a section number.  The
references are numbered in the order in which they are cited.
Examples of the format to be followed are given at the end of this
document.

The reference list at the end of this document is created using
BibTeX, which looks through the file {\tt report.bib} for the entries
cited in the LaTeX source file.  The format of the reference list is
determined by the bibliography style file {\tt spiebib.bst}, as
specified in the \verb|\bibliographystyle{spiebib}| command.
Alternatively, the references may be directly formatted in the LaTeX
source file.

For books\cite{Lamport94,Alred03,Goossens97} the listing includes the
list of authors, book title (in italics), page or chapter numbers,
publisher, city, and year of publication.  A reference to a journal
article\cite{Metropolis53} includes the author list, title of the
article (in quotes), journal name (in italics, properly abbreviated),
volume number (in bold), inclusive page numbers, and year.  By
convention\cite{Lamport94}, article titles are capitalized as
described in Sec.~\ref{sec:title}.  A reference to a proceedings paper
or a chapter in an edited book\cite{Gull89a} includes the author list,
title of the article (in quotes), conference name (in italics), if
appropriate, editors, volume or series title (in italics), volume
number (in bold), if applicable, inclusive page numbers, publisher,
city, and year.  References to an article in the SPIE Proceedings may
include the conference name, as shown in Ref.~\citenum{Hanson93c}.

Citations to the references are made using superscript numerals, as
demonstrated in the preceding paragraph.  One may also directly refer
to a reference within the text, e.g., ``as shown in
Ref.~\citenum{Metropolis53} ..."

%% -----------------------------------------------------------
\subsection{Footnotes}
Footnotes\footnote{Footnotes are indicated as superscript symbols to
  avoid confusion with citations.} may be used to provide auxiliary
information that doesn't need to appear in the text, e.g., to explain
measurement units.  They should be used sparingly, however.

Only nine footnote symbols are available in LaTeX. If you have more
than nine footnotes, you will need to restart the sequence using the
command \verb|\footnote[1]{Your footnote text goes here.}|. If you
don't, LaTeX will provide the error message {\tt Counter too large.},
followed by the offending footnote command.

%%%%%%%%%%%%%%%%%%%%%%%%%%%%%%%%%%%%%%%%%%%%%%%%%%%%%%%%%%%%%
\section{SECTION FORMATTING} \label{sec:sections}

Section headings are centered and formatted completely in uppercase
11-point bold font.  Sections should be numbered sequentially,
starting with the first section after the Abstract.  The heading
starts with the section number, followed by a period.  In LaTeX, a new
section is created with the \verb|\section{}| command, which
automatically numbers the sections.

Paragraphs that immediately follow a section heading are leading
paragraphs and should not be indented, according to standard
publishing style\cite{Lamport94}.  The same goes for leading
paragraphs of subsections and sub-subsections.  Subsequent paragraphs
are standard paragraphs, with 14-pt.\ (5 mm) indentation.  An extra
half-line space should be inserted between paragraphs.  In LaTeX, this
spacing is specified by the parameter \verb|\parskip|, which is set in
{\tt spie.cls}.  Indentation of the first line of a paragraph may be
avoided by starting it with \verb|\noindent|.
 
%% -----------------------------------------------------------
\subsection{Subsection Attributes}

The subsection heading is left justified and set in 11-point, bold
font.  Capitalization rules are the same as those for book titles.
The first word of a subsection heading is capitalized.  The remaining
words are also capitalized, except for minor words with fewer than
four letters, such as articles (a, an, and the), short prepositions
(of, at, by, for, in, etc.), and short conjunctions (and, or, as, but,
etc.).  Subsection numbers consist of the section number, followed by
a period, and the subsection number within that section.

%% -----------
\subsubsection{Sub-subsection attributes}
The sub-subsection heading is left justified and its font is 10 point,
bold.  Capitalize as for sentences.  The first word of a
sub-subsection heading is capitalized.  The rest of the heading is not
capitalized, except for acronyms and proper names.

%%%%%%%%%%%%%%%%%%%%%%%%%%%%%%%%%%%%%%%%%%%%%%%%%%%%%%%%%%%%%
\section{FIGURES AND TABLES}

Figures are numbered in the order of their first citation.  They
should appear in numerical order and on or after the same page as
their first reference in the text.  Alternatively, all figures may be
placed at the end of the manuscript, that is, after the Reference
section.  It is preferable to have figures appear at the top or bottom
of the page.  Figures, along with their captions, should be separated
from the main text by at least 0.2 in.\ or 5 mm.

Figure captions are centered below the figure or graph.  Figure
captions start with the figure number in 9-point bold font, followed
by a period; the text is in 9-point normal font; for example,
``{\footnotesize{Figure 3.}  Original image...}".  See
Fig.~\ref{fig:example} for an example of a figure caption.  When the
caption is too long to fit on one line, it should be justified to the
right and left margins of the body of the text.

Tables are handled identically to figures, except that their captions
appear above the table.
%% Use following command to specify that graphics file is in a
%% directory other than this LaTeX source file.  Note use of / to
%% separate subdirectories, for UNIX and Windows OS.
%% \graphicspath{{H:/HANSON/SPIESTY/}} tabular environment useful for
%% creating an array of images
% -------------
\begin{figure}
  \begin{center}
    \begin{tabular}{c}
      \includegraphics[height=7cm]{naive.jpg}
    \end{tabular}
  \end{center}
  \caption[example]
  % >>>> use \label inside caption to get Fig. number with \ref{}
  { \label{fig:example} Figure captions are used to describe the
    figure and help the reader understand it's significance.  The
    caption should be centered underneath the figure and set in
    9-point font.  It is preferable for figures and tables to be
    placed at the top or bottom of the page. LaTeX tends to adhere to
    this standard.}
\end{figure}
% -------------

For further details concerning the inclusion of grayscale and color
images, consult SPIE's {\it Author Guide for Publication and
  Presentation}.
 
%%%%%%%%%%%%%%%%%%%%%%%%%%%%%%%%%%%%%%%%%%%%%%%%%%%%
\appendix    %>>>> this command starts appendixes
%%%%%%%%%%%%%%%%%%%%%%%%%%%%%%%%%%%%%%%%%%%%%%%%%%%%
\section{MISCELLANEOUS FORMATTING DETAILS} \label{sec:misc}

It is often useful to refer back (or forward) to other sections in the article.  Such references are made by section number.  When a section reference starts a sentence, Section is spelled out; otherwise use its abbreviation, for example, ``In Sec.~2 we showed..." or ``Section~2.1 contained a description...".  References to figures, tables, and theorems are handled the same way.

At the first occurrence of an acronym, spell it out, followed by the acronym in parentheses, e.g., noise power spectrum (NPS).  
 
%%-----------------------------------------------
\subsection{Formatting Equations} 
Equations may appear in line with the text, if they are simple, short, and not of major importance; e.g., $\beta = b/r$.  Important equations appear on their own line.  Such equations are centered.  For example, ``The expression for the minus-log-posterior is
	\begin{equation}
	\label{eq:alpha}
\phi = |{\rm\bf y} - {\rm\bf A}{\rm\bf x}|^2 + \alpha \log p({\rm\bf x}) \, ,
	\end{equation}
where $\alpha$ determines the strength of ..."  Principal equations are numbered, with the equation number placed within parentheses and right justified.  

Equations are considered to be part of a sentence and should be punctuated accordingly. In the above example, a comma follows the equation because the next line is a subordinate clause.  If the equation ends the sentence, a period should follow the equation.  The line following an equation should not be indented unless it is meant to start a new paragraph.  Indentation after an equation is avoided in LaTeX by not leaving a blank line between the equation and the subsequent text.

References to equations include the equation number in parentheses, for example, ``Equation~(\ref{eq:alpha}) shows ..." or ``Combining Eqs.~(2) and (3), we obtain..."  Using a tilde in the LaTeX source file between two characters avoids unwanted line breaks.

%%-----------------------------------------------------------
\subsection{Formatting Theorems} 

To include theorems in a formal way, the theorem identification should appear in a 10-point, bold font, left justified and followed by a period.  The text of the theorem continues on the same line in normal, 10-point font.  For example, 

\noindent{\bf Theorem 1.} For any unbiased estimator...

Formal statements of lemmas and algorithms receive a similar treatment.

%%%%%%%%%%%%%%%%%%%%%%%%%%%%%%%%%%%%%%%%%%%%%%%%%%%%
\section{SOME LATEX GUIDANCE} \label{sec:latex}

%%-----------------------------------------------------------
\subsection{Margins and PostScript Fonts}
 
Manuscripts submitted electronically to as PostScript (PS) files must have the correct margins. LaTeX margins are related to the document's paper size. The paper size is set at two separate places in the process of creating a PS file. The first place is in {\tt latex}. The default in {\tt article.tex}, on which {\tt spie.cls} is based, is USA letter paper. To format a document for A4 paper, the first line of the LaTeX source file should be \verb|\documentclass[a4paper]{spie}|.   

The output of the LaTeX utility is a file with the extension DVI (for Device Independent), which encodes the formatted document.  The application DVIPS is typically used to convert the DVI file to a PS file.  DVIPS has its own default paper size, which can be overridden with the option ``{\tt -t letter}" or ``{\tt -t a4}".  
If the foregoing steps do not produce the correct top margin, you can move the text lower on the page (by 9 mm) with the command \verb|\addtolength{\voffset}{9mm}|, placed right after the \verb|\documentclass| command, for example.

Another DVIPS option specifies the incorporation of (scalable) PostScript Type 1 fonts in its output PS file. This feature is important for obtaining a subsequent PDF file that will be clearly displayed on a computer monitor by Adobe Acrobat Reader.  The option ``{\tt -P pdf}" makes DVIPS include these fonts in its output PS file.

%%-----------------------------------------------------------
\subsection{Bold Math Symbols} 

The math package from the American Mathematical Society allows one to easily produce bold math symbols, well beyond what is available in LaTeX. It also provides many useful capabilities for creating elaborate mathematical expressions. You need to load the AMS math package near the top of the LaTeX source file, right after the \verb+\documentclass+ command:\\[1ex]
\verb+\usepackage[]{amsmath}+ \\[1ex]
Then for bold math symbols use \verb+\boldsymbol+ in equations, e.g., 
\verb+$\boldsymbol{\pi}$+ 
yields a bold pi.  You can make it easier to use by defining a command:\\[1ex]
\verb+\newcommand{\bm}[1]{\boldsymbol{#1}}+ \\[1ex]
and then using it like so \verb+$\bm{\pi}$+.

Not all math symbols are available in bold.  In a pinch, you can use \verb+\pmb+ ("poor man's bold"), which is defined in \verb+amsmath+. This command approximates a bold character with a superposition of several, slightly displaced unbold characters.

If you want a Greek symbol in the article title, it should be both larger and bold. The easiest thing is to load the AMS math package as described above. 
Then, in the title, use something like:\\[1ex]
\verb+\title{Estimation of {\LARGE$\boldsymbol\alpha$} by a Monte Carlo technique}+ \\[1ex]
Note that the command to create the alpha character is enclosed within braces to form a self-contained environment. The use of \verb+\LARGE+ in this example may not be needed when using nondefault font packages, such as the {\tt times} package, because of how the article title is handled in {\tt spie.cls}.

%%-----------------------------------------------------------
\subsection{Uppercase letters and special symbols in BibTex} 

BibTeX tries to enforce standard publishing rules regarding article titles and authors' names; it sometimes changes uppercase letters to lower case. BibTeX also has trouble with umlauts, generally created in LaTeX with \verb+\"{o}+, because it is looking for the \verb+"+ to end the input line. 

The general rule for overriding LaTeX's and BibTex's reinterpretation of your input text is to put the items you wish to be unchanged in braces. Thus, to obtain an umlaut in an author's name or in an article title, or to force an uppercase letter, do something like the following: \\[1ex]
\verb+ @article{Kaczmarz37,+ \\ 
\verb+ author = "S. Kaczmarz",+ \\ 
\verb+ title  = "Angen{\"{a}}hrte {A}ufl{\"{o}}sung von {S}ystemen linearer {G}leichungen",+ \\ 
\verb+ journal= "Bull. Acad. Polon. Sci. Lett.",+ \\ 
\verb+ volume = "A35",+ \\ 
\verb+ pages  = "355-357",+ \\	
\verb+ year   = "1937"	} + \\[1ex]
This example shows the use of both umlauts and uppercase letters.
%%%%%%%%%%%%%%%%%%%%%%%%%%%%%%%%%%%%%%%%%%%%%%%%%%%%%%%%%%%%%
\acknowledgments     %>>>> equivalent to \section*{ACKNOWLEDGMENTS}       
 
This unnumbered section is used to identify those who have aided the authors in understanding or accomplishing the work presented and to acknowledge sources of funding.  

%%%%%%%%%%%%%%%%%%%%%%%%%%%%%%%%%%%%%%%%%%%%%%%%%%%%%%%%%%%%%
%%%%% References %%%%%

\bibliography{report}   %>>>> bibliography data in report.bib
\bibliographystyle{spiebib}   %>>>> makes bibtex use spiebib.bst

\end{document} 
