%  article.tex (Version 3.3, released 19 January 2008)
%  Article to demonstrate format for SPIE Proceedings
%  Special instructions are included in this file after the
%  symbol %>>>>
%  Numerous commands are commented out, but included to show how
%  to effect various options, e.g., to print page numbers, etc.
%  This LaTeX source file is composed for LaTeX2e.

%  The following commands have been added in the SPIE class 
%  file (spie.cls) and will not be understood in other classes:
%  \supit{}, \authorinfo{}, \skiplinehalf, \keywords{}
%  The bibliography style file is called spiebib.bst, 
%  which replaces the standard style unstr.bst.  

\documentclass[]{spie}  %>>> use for US letter paper
%%\documentclass[a4paper]{spie}  %>>> use this instead for A4 paper
%%\documentclass[nocompress]{spie}  %>>> to avoid compression of citations
%% \addtolength{\voffset}{9mm}   %>>> moves text field down
%% \renewcommand{\baselinestretch}{1.65}   %>>> 1.65 for double spacing, 1.25 for 1.5 spacing 
%  The following command loads a graphics package to include images 
%  in the document. It may be necessary to specify a DVI driver option,
%  e.g., [dvips], but that may be inappropriate for some LaTeX 
%  installations. 
\usepackage{graphicx}
\usepackage{subfig}
\usepackage{amsmath}
\usepackage{amssymb}
\usepackage{hyperref}

\title{Texture mapping 3D planar models of indoor environments with noisy camera poses} 

%>>>> The author is responsible for formatting the 
%  author list and their institutions.  Use  \skiplinehalf 
%  to separate author list from addresses and between each address.
%  The correspondence between each author and his/her address
%  can be indicated with a superscript in italics, 
%  which is easily obtained with \supit{}.

\author{Peter Cheng, Michael Anderson, Stewart He, Avideh Zakhor
\skiplinehalf
University of California, Berkeley\\
}

 

%%%%%%%%%%%%%%%%%%%%%%%%%%%%%%%%%%%%%%%%%%%%%%%%%%%%%%%%%%%%% 
%>>>> uncomment following for page numbers
% \pagestyle{plain}    
%>>>> uncomment following to start page numbering at 301 
%\setcounter{page}{301} 
 
\begin{document}
\maketitle

%%%%%%%%%%%%%%%%%%%%%%%%%%%%%%%%%%%%%%%%%%%%%%%%%%%%%%%%%%%%% 
\begin{abstract}
  Automated 3D modeling of building interiors is used in applications
  such as virtual reality and environment mapping. Texturing these
  models allows for photorealistic visualizations of the data
  collected by such modeling systems. Camera poses obtained by these
  systems and used for texturing often suffer from inaccuracies
  however, resulting in visible discontinuities when successive images
  are projected onto a surface for texturing. Existing methods to
  stitch images together are often computationally expensive and work
  independently of pose estimates and geometry information. We propose
  an efficient method to refine camera poses in 2D using both existing
  estimates and geometry information, followed by two different
  methods to composite images together, based on the geometry of
  surfaces being textured. The effectiveness of our methods is
  demonstrated on a number of different indoor environments.
\end{abstract}

% >>>> Include a list of keywords after the abstract

\keywords{Texture Mapping, Reconstruction, Image Stitching, Mosaicing}

%%%%%%%%%%%%%%%%%%%%%%%%%%%%%%%%%%%%%%%%%%%%%%%%%%%%%%%%%%%%%
\section{Introduction}
\label{sec:introduction} % \label{} allows reference to this section

Three-dimensional modeling of indoor environments has a variety of
applications such as training and simulation for disaster management,
virtual heritage conservation, and mapping of hazardous sites. Manual
construction of these digital models can be time-consuming, and as a
result, automated 3D site modeling has garnered much interest in
recent years.

The first step in automated 3D modeling is the physical scanning of an
environment's geometry. An indoor modeling system must recover its own
poses within an environment while simultaneously reconstructing the 3D
structure of the environment itself \cite{chen2010indoor,
  liu2010indoor, kua2012loopclosure}. This is known as the
simultaneous localization and mapping (SLAM) problem, and is generally
solved by taking readings from laser range scanners, cameras, and
inertial measurement units (IMUs) at multiple locations within the
environment.

In this paper, we work with data obtained from a backpack-mounted
system, carried by an ambulatory human. Such a system provides
advantages over more common wheel-mounted systems in terms of agility
and portability, but results in higher difficulty and error when
performing localization \cite{liu2010indoor}. As a result, approaches
to texture mapping that rely on accurate camera poses produce poor
results, leading to the development of the methods contained in this
paper. Besides localization information and images, environment
geometry is required as well for texture mapping. In this paper we
work with low-resolution models obtained by fitting planar surfaces to
point clouds generated by the backpack system \cite{sanchez2012point}.



SOME CONCLUSION TO THE INTRODUCTION GOES HERE Though our texture
mapping procedure was designed with the aforementioned system in mind,
it is applicable to other systems, provided the required inputs of
planar geometry, images with rough extrinsic pose estimates, and
either known or estimated intrinsic camera calibration matrices. An
important benefit of our method is its relatively low complexity and
runtime, and independence of many steps, allowing for efficient tuning
of parameters and easy accomodation of new input data given updates in
localization or model-generation algorithms.

The remainder of the paper is organized as follows. Section
\ref{sec:simpleTextureMapping} explains how input images are projected
onto our geometry and a simple approach towards texturing. Section
\ref{sec:existingApproaches} covers more sophisticated existing
approaches, using image matching and stitching, and their performance
on our datasets. Section \ref{sec:2dAlignment} contains our approach
towards image alignment, followed by Section
\ref{sec:imageCompositing}, which describes our two methods of
compositing images. Section \ref{sec:resultsAndConclusions} contains
results and conclusions.
%%%%%%%%%%%%%%%%%%%%%%%%%%%%%%%%%%%%%%%%%%%%%%%%%%%%%%%%%%%%%

\section{Background}
\label{sec:background}
\subsection{Existing Approaches}
\label{sec:existingApproaches}
\subsection{

  \section{Simple Texture Mapping}
  \label{sec:simpleTextureMapping}

  The geometry of the texture mapping process for a planar surface is
  shown in Figure \ref{fig:projection}.  As described earlier, we are
  provided with a set of $M$ images to texture a target plane. Each
  image has a camera matrix $P_i$ for $i=1..M$, which translates a 3D
  point in the world coordinate system to a 2D point or pixel in image
  $i$'s coordinates. A camera matrix $P_i$ is composed of the camera's
  intrinsic parameters, containing focal length and image center, as
  well as extrinsic parameters which specify the rotation and
  translation of the camera's position in 3D world coordinates at the
  time that image $i$ is taken. These extrinsic parameters are
  determined by the backpack hardware and localization algorithms
  \cite{chen2010indoor, liu2010indoor, kua2012loopclosure} and are
  quite noisy.

  Because our backpack system takes photos at a rate of 5 Hz,
  thousands of images are available for texturing each surface in our
  model. Our goal in designing a texture mapping process is to decide
  which of these images should be used, and where their contents
  should map onto the final texture, in order to minimize any visual
  discontinuities or seams that would suggest that the plane's texture
  is not composed of a single continuous image.

  \begin{figure}
    \begin{minipage}[b]{0.45\linewidth}
      \centering
      \includegraphics[height=1.5in]{Projection.pdf}
      \caption{Planes are specified in 3D space by four corners $C_1$
        to $C_4$. Images are related to each plane through the camera
        matrics $P_{1..m}$. }
      \label{fig:projection}
    \end{minipage}
    \hspace{0.5cm}
    \begin{minipage}[b]{0.45\linewidth}
      \centering
      \includegraphics[height=1.125in]{scoringFunction.jpg}
      \caption{We minimize camera angle $\alpha$ and distance $d$ by
        maximizing the scoring function $\frac{1}{d} (-1 \cdot
        \vec{c}) \cdot \vec{n}$}
      \label{fig:scoringFunction}
    \end{minipage}
  \end{figure}


  \subsection{Tile-based Texture Mapping}
  \label{sec:tileBasedMapping}
  Ignoring the fact that the camera matrices $P_{1..M}$ are
  inaccurate, a simple texturing approach can be performed by
  discretizing the target plane into small square tiles, in our case 5
  pixels across, and choosing an image to texture each tile.

  We choose to work with rectangular units to ensure that borders
  between any two distinct images in the final texture are either
  horizontal or vertical. Since most strong environmental features
  inside buildings are horizontal or vertical, any visible seams in
  our texture thus intersect them minimally and are less noticeable.

  In order to select an image for texturing a tile $t$, we must first
  gather a list of candidate images that contain all four of its
  corners, which we can quickly check by projecting $t$ into each
  image using the $P_i$ camera matrices. Furthermore, each candidate
  image must have been taken at a time when its camera had a clear
  line-of-sight to $t$, which can be calculated using standard
  ray-polygon intersection tests between the camera location, $t$, and
  every other plane \cite{rayintersection}.

  Once we have a list of candidate images for $t$, we define a scoring
  function in order to objectively select the best image for texturing
  $t$. Since resolution decreases and camera pose errors compound over
  distance, we wish to minimize the distance between cameras and the
  surfaces they texture. Additionally, we desire images that are
  projected perpendicularly onto the plane, maximizing the resolution
  and amount of useful texture available in their projections, as well
  as minimizing any parallax effects due to real-world geometry not
  accurately represented by our digital model. In other words, we wish
  to minimize the angle between the tile's normal vector and the
  camera axis for images selected for texture mapping. These two
  criteria can be met by maximizing the function $\frac{1}{d} (-1
  \cdot \vec{c}) \cdot \vec{n}$ as shown in Figure
  \ref{fig:scoringFunction}. Specifically, $d$ is the distance between
  the centers of a camera and a tile, and $\vec{n}$ and $\vec{c}$ are
  the directions of the plane's normal and the camera axis
  respectively.

\begin{figure}
  \centering \subfloat[][]{\includegraphics[width=3.4in,
    height=1.2in]{wall2_naive.jpg}}
  \subfloat[][]{\includegraphics[width=3.4in,
    height=1.2in]{wall2_naive_shift.jpg}}

  \centering \subfloat[][]{\includegraphics[width=3.4in,
    height=1.2in]{wall2_cache_shift.jpg}}
  \subfloat[][]{\includegraphics[width=3.4in,
    height=1.2in]{wall2_shortest_shift.pdf}}

  \centering \subfloat[][]{\includegraphics[width=3.4in,
    height=1.2in]{wall2_cache_shift_details.png}}
  \subfloat[][]{\includegraphics[width=3.4in,
    height=1.2in]{wall2_shortest_shift_details.png}}

  \centering \subfloat[][]{\includegraphics[width=3.4in,
    height=1.2in]{wall2_cache_shift_blend.pdf}}
  \subfloat[][]{\includegraphics[width=3.4in,
    height=1.2in]{wall2_shortest_shift_blend.pdf}}

  \caption{(a): Tile-based texturing. (b): Tile-based texturing after
    image alignment. (c): Tile-based texturing after image alignment
    with caching. (d): Shortest path texturing after image
    alignment). (e,f): Comparison of image artifacts in (c)
    vs. (d). (g,h): Blending applied to (c) and (d).}
  \label{fig:compareAll}
\end{figure}


As Figure \ref{fig:compareAll}(a) demonstrates, this approach leads to
the best texture for each tile independently, but results in many
image boundaries with abrupt discontinuities between tiles, due to
significant misalignment between images. Given our high number of
input images, a better image selection procedure could be used, as
described in Section \ref{sec:imageCompositing}, but a more
significant and reliable improvement can be found through the
alignment of our images.

\section{Existing Approaches to Image Alignment}
\label{sec:existingApproaches}
Stitching together multiple images to produce a larger image is a
commonly performed task, with many successful approaches over the past
few decades. Parts of images are first matched to eachother, through
direct pixel comparisons, or more commonly through feature detection
and matching. Images are then adjusted to maximize matches, either by
calculating homographies between pairs of images, or by modifying
camera poses in 1 to 6 degrees of freedom.

Feature detection and matching works best when multiple unique visual
references exist in the environment and are present within multiple
images. In contrast, our indoor environments have a high prevalence of
bare surfaces, as well as repeating textures that cause difficulty in
disambiguating features. This lack of reliable reference points often
results in errors when matching images together.

Additionally, our datasets often contain long chains of images, which
leads to error accumulation when image correspondences are not
accurate. For example, when matching a long chain of images through
homographies, a pixel in the $nth$ image is translated into the first
image's coordinates by multiplying by the $3\times3$ matrices $H_1 H_2
H_3 ... H_n$. Errors in any of these homography matrices are
propagated to all further images, resulting in drift. Figure
\ref{fig:mosaic3D}(a) shows the output of the AutoStitch software
package, which performs homography-based image mosaicing
\cite{autostitch}. Though AutoStitch performs well in areas with dense
features, it has difficulty with even short segments of bare walls,
and produces errors that cause straight lines to appear
distorted. Many areas with fewer visual features or repeating texture
patterns failed outright using AutoStitch.

\begin{figure}
  \centering
  \subfloat[][]{\includegraphics[width=6in]{autostitchResult.jpg}}

  \centering
  \subfloat[][]{\includegraphics[width=6in]{graphApproach.jpg}}

  \centering \subfloat[][]{\includegraphics[width=6in]{finalLong.jpg}}

  \caption{Texture alignment via (a) image mosaicing, (b) the
    graph-based localization refinement algorithm from
    \cite{chen2010indoor}, and (c) the method proposed in this paper.}
  \label{fig:mosaic3D}
\end{figure}

Image projections can also be aligned by iteratively adjusting camera
poses to maximize matches. This is generally done over 6 degrees of
freedom \cite{liu2010indoor}. The approach in [2] works well locally,
but over larger areas results again in error accumulation and drift,
as shown in Figure \ref{fig:mosaic3D}(b). Furthermore, it adjusts
image locations using a long iterative process, which results in high
runtime.


\section{2D Image Alignment with Geometry Information}
\label{sec:2dAlignment}
In this section, we describe our method of efficient and robust image
alignment. Our approach consists of three main parts. First, all
images are projected onto the surface and lines within these
projections are detected. These lines are then matched up with
geometry-based lines comprising the surface's boundary and
intersection with other surfaces. Second, occlusion checks are
performed to remove invalid parts of each image for the target
surface. These two steps are image-independent and performed in
parallel. Third, we detect SIFT feature matches between pairs of
images and solve a linear least squares problem in 2D to maximize
matches.

All of our following calculations and alignments are performed in 2D,
partly for efficiency, partly because we have a large number of images
to choose from, and partly because the nature of our input data is
such that localization error chiefly occurs in two dimensions, in the
same plane as the surface being projected onto. Recall that our data
acquisition system is backpack-mounted, with cameras facing to the
sides of the operator. The operator makes efforts to walk parallel to
walls, and the localization and model-generation algorithms that
provide our input conform the operator's path to be straight and
parallel to detected surfaces \cite{kua2012loopclosure,
  sanchez2012point}. Since the operator walks in as upright a position
as possible, errors in roll, yaw, and the operator's distance from
parallel surfaces are minimal. Thus, our highest errors stem from
uncertainty in the operator's pitch, equivalent to rotation around the
camera axes, as well as location along surfaces. These equate to 2D
rotation and translation in a surface's plane.

\subsection{Geometry-based Alignment}
\label{sec:geometryAlignment}
After computing each image's projection onto the target surface, as
described in Section \ref{sec:simpleTextureMapping}, we detect line
segments in the image projections using Hough transforms. Experience
and intuition show that walls in indoor environments often contain
linear features that are either horizontal or vertical, corresponding
to doors, windows, posters, etc. Thus, when texturing walls, we rotate
images in 2D such that dominant lines are made to be horizontal or
vertical. This can be further extrapolated by orienting lines with a
wall's boundaries, for example in areas with a slanted roof.

\begin{figure}
  \centering
  \subfloat[][]{\includegraphics[height=1.5in]{geometryAlign_planes.png}}
  \subfloat[][]{\includegraphics[height=1.5in]{geomAlign}}
  \caption{(a) When texturing the red surface, it makes sense to align
    images to surface boundaries and intersections with other
    surfaces. (b) The darker lines are geometry-based lines, while
    lighter lines are lines detected in the image via Hough
    transform. Above is the original projection using the input camera
    poses, and below is the projection after rotation and translation
    for alignment.}
  \label{fig:geometryAlignment}
\end{figure}


At this point in time, image occlusions have not been accounted
for. As a result, some image projections contain texture that should
project to an adjacent surface, generally with a linear boundary where
the two surfaces meet. If this linear boundary is detected by Hough
transform, the image is rotated and shifted in 2D such that the visual
boundary between two surfaces in an image projection matches the
physical boundary in our digital model. An example of such an
adjustment is in Figure \ref{fig:geometryAlignment}

To perform the rotation and alignment, we use the RANSAC
\cite{fischler1981random} framework. RANSAC allows us to determine an
optimal rotation and translation to match up two groups of line
segments, while accounting for outliers. First, a rotation angle is
computed by gathering all pairs of image line segments and geometry
line segments with less than 20 degrees between them. We then run the
RANSAC algorithm by sampling 4 of these angles at a time, and using
their average as the fitting function. Angles over 2 degrees from this
average are considered outliers. Once this 2d rotation angle is
determined, it is applied to the image, and projections are
re-calculated and image-based lines are re-detected. Now, all pairs of
image line segments and geometry line segments with less than 0.2
degree difference and less than 250 mm distance at their furthest
points are gathered. RANSAC is then similarly employed to calculate
optimal horizontal and vertical translations independently, by
sampling 4 pairs at a time and considering an outlier threshold of 10
mm. If a translation is applied, we indicate that the image was
aligned to geometry in one or both dimensions, which will reduce
further shifting in Section \ref{sec:robustSIFTFeatureMatching}.

\subsection{Image Occlusion}
\label{sec:imageOcclusion}
Now that images have been aligned to geometry where possible, we
perform a simple recursive occlusion procedure on each image. This is
done by performing the intersection tests in section
\ref{sec:simpleTextureMapping} in a regularly spaced grid. Where four
corners of a rectangular region are occluded, texture is
removed. Where no corners are occluded, nothing occurs. Where there is
a mixture of both, the rectangular region is subdivided into four, and
the same process is performed on each. By performing Section
\ref{sec:geometryAlignment}'s alignment procedure before occlusion,
texture belonging to other surfaces is accurately removed, which is
necessary for the next section.

\subsection{2D Feature Alignment}
\label{sec:robustSIFTFeatureMatching}
Our next step is to align overlapping images by searching for
corresponding points between all pairs of overlapping images. We use
SIFT features for their high detection rate, and choose to use feature
alignment rather than pixel or intensity-based alignment due to the
differences in lighting as well as possible occlusion among our
images, both of which feature alignment is less sensitive to
\cite{lowe1999object, mikolajczyk2005performance, szeliski2006image}.
We use SiftGPU \cite{siftgpu} for its high performance on both feature
detection as well as pairwise matching. These matches determine $d^x$
and $d^y$ distances between each pair of features for two image
projections, though these distances may not always be the same for
different features. Since indoor environments often contain repetitive
features such as floor tiles or doors, we need to ensure that
SIFT-based distances are reliable. First, we only perform alignment on
parts of images that overlap given the original noisy poses. Second,
we discard feature matches that correspond to an image distance
greater than 40 pixels from what the noisy poses estimate. In order to
utilize the remaining feature matches robustly, RANSAC
\cite{fischler1981random} is again used to estimate the optimal
$d^x_{i,j}$ and $d^y_{i,j}$ distances between two images $i$ and
$j$. For this application, the RANSAC fitting function finds the
average distance between selected matches in a pair of images, and the
distance function for a pair of points is chosen to be the difference
between those points' SIFT match distance and the average distance
computed by the fitting function. We use a 10 pixel threshold, so that
SIFT matches are labeled as outliers if their horizontal or vertical
distances are not within 10 pixels of the average distance computed by
the fitting function.

We now use the RANSAC-calculated $d^x_{i,j}$ and $d^y_{i,j}$ distances
between each pair of images to refine their positions using weighted
linear least squares. There are a total of $M^{2}$ possible pairs of
images, though we only generate distances between images with SIFT
matches. Given these distances and the original image location
estimates, we can solve a least squares problem
($\textrm{min}_{\vec{\beta}} ||A \vec{\beta} - \vec{\gamma}||_2^2 $)
to estimate the location of the images on the plane. The
$M$-dimensional vector $\vec{\beta}$ represents the unknown $x$
location of each image on the plane for $1 \dots M$. The optimal $x$
and $y$ locations are obtained in the same way, so we only consider
the $x$ locations here:

\[\vec{\beta} =
\begin{pmatrix}
  x_1, & x_2, & x_3, & \cdots & x_{M-1}, & x_M
\end{pmatrix}
\]

The $N \times M$ matrix $A$ is constructed with one row for each pair
of images with measured distances produced by the SIFT matching
stage. A row in the matrix has a $-1$ and $1$ in the columns
corresponding to the two images in the pair. For example, the matrix
below indicates a SIFT-based distance between images 1 and 2, images 1
and 3, images 2 and 3, etc.
\[
A =
\begin{pmatrix}
  -1 & 1 & 0 & \cdots & 0 & 0\\
  -1 & 0 & 1 & \cdots & 0 & 0\\
  0 & -1 & 1 & \cdots & 0 & 0\\
  \vdots  & \vdots & \vdots & \ddots & \vdots  & \vdots\\
  0 & 0 & 0 & \cdots & 1 & 0 \\
  0 & 0 & 0 & \cdots & -1 & 1 \\
  1 & 0 & 0 & \cdots & 0 & 0\\
\end{pmatrix}
\]
If only relative distances between images are included, the absolute
location of the images can not be calculated, and the matrix is rank
deficient. From Section \ref{sec:geometryAlignment}, a number of
images are anchored to geometry points, in one or both dimensions, and
thus their locations can be used to fix the rest in place. In case no
anchor images exist, we simply arbitrarily pick an image, as we have
no other reference points to work with. In the above matrix, the first
image is set to be such an anchor, simply by placing a $1$ in its
column.

The $N$-dimensional observation vector $\vec{\gamma}$ is constructed
using the SIFT-based distances generated in the RANSAC matching
stage. Elements in the observation vector corresponding to anchor
images are simply their locations as determined by the original noisy
localization. Thus a $\vec{\gamma}$ corresponding to the above matrix
can be written as:

\[
\vec{\gamma}^T =
\begin{pmatrix}
  d_{1,2}, &d_{1,3}, &d_{2,3}, &\hdots &d_{N-2,N-1}, &d_{N-1,N}, &x_1
\end{pmatrix}
\]

The $\vec{\beta}$ that minimizes $||A \vec{\beta} -
\vec{\gamma}||_2^2$ results in a set of image locations on the plane
that best honors all the SIFT-based distance measurements between
images. This solution however does not make use of our noisy camera
poses, and will fail when no SIFT matches are found between one
segment of the plane and another. To account for this, we add rows to
the $A$ matrix and observations to the $\vec{\gamma}$ vector
corresponding to the original noisy distances. We then add weighting
to our least squares problem where the SIFT distances and anchor
values are given a high weight e.g. 1, while the noisy distances are
given a smaller weight e.g. 0.01.

After completing this same process for the $y$ dimension as well, and
making the resultant shifts, our images overlap and match each other
with far greater accuracy. Applying the simple mapping scheme in
Section \ref{sec:tileBasedMapping} to the same wall used in that
section results in Figure \ref{fig:compareAll}(b), which has far fewer
discontinuities, though errors due to lighting differences and
repeating features are still visible.

\section{Image Compositing}
\label{sec:imageCompositing}
In this section, we revisit the tile-based texturing approach from
Section \ref{sec:simpleTextureMapping}, with an added caching
mechanism to reduce image boundaries. This method works well given all
manner of camera poses and surfaces, but for optimal cases where we
have large sections of usable texture from images, we propose a
superior method that further reduces image boundaries.

\subsection{Tile-Mapping with Caching}
\label{sec:mappingWithCaching}
From Section \ref{sec:tileBasedMapping}, we saw that discontinuities
occur where adjacent tiles are textured by different images. Though
Section \ref{sec:2dAlignment}'s image alignment removes many such
discontinuities, there are still cases where seams are visible due to
imprecise matching or other factors such as model-based errors. To
reduce the cases where this happens, it makes sense to take into
account image selections made by neighboring tiles while texture
mapping a given tile. By using the same image across tile boundaries,
we can eliminate a discontinuity altogether. If this is not possible
because a tile is not visible in images used by neighboring tiles,
using similar images across tile boundaries also leads to less
noticeable discontinuities.

Essentially a caching mechanism, we select the best image for a tile
$t$ by searching through two subsets of images for a viable candidate,
before searching through the entire set of valid images. The first
subset of images is the images selected by adjacent tiles that have
already been textured. We must first check which of these images can
map to $t$, and then of those, we make a choice according to the
scoring function in Figure \ref{fig:scoringFunction}. Before reusing
this image, we ensure it meets the criteria $\alpha < 45^\circ$, in
order to ensure a high resolution projection, with $\alpha$ as the
camera angle as shown in Figure \ref{fig:scoringFunction}.

If no satisfactory image is found in the first subset, we check a
second subset of images, consisting of those taken near the ones in
the first subset, both spatially and temporally. These images are not
the same as the ones used for neighboring tiles, but are taken at a
similar location and time, suggesting that their localization and
projection are quite similar, and thus likely matched more cleanly. If
no viable image is found according to the same criteria as before, we
search the entire set of candidate images, selecting based on the same
scoring function from Figure \ref{fig:scoringFunction}.

The result of this caching approach is shown in Figure
\ref{fig:compareAll}(c), where seams are now reduced as compared to
Figure \ref{fig:compareAll}(b). However, some discontinuities are
still present, as visible in the many posters on the wall with breaks
in their borders.

\begin{figure}
  \centering
  \subfloat[][]{\includegraphics[width=1in]{projectionWall.pdf}}
  \centering
  \subfloat[][]{\includegraphics[width=1.1in]{projectionCeiling.pdf}}
  \centering
  \subfloat[][]{\includegraphics[width=0.9in]{projectionWallCrop.pdf}}
  \caption{(a) Images for vertical planes are tilted, but their camera
    axes are more or less normal to their respective planes. (b)
    Camera axes for ceiling images are at large angles with respect to
    plane normals. (c) Wall images are cropped to be rectangular.}
  \label{fig:projectionAngles}
\end{figure}


As mentioned earlier, our data comes from a mobile backpack
system. Human operators can not carry the backpack in a perfectly
upright position and are bent forwards at 15 to 20 degrees with
respect to the vertical direction. As a result, cameras facing
sideways are head on with respect to vertical walls, while cameras
oriented towards the top or bottom of the backpack are at an angle
with respect to horizontal floors and ceilings. This is depicted in
Figures \ref{fig:projectionAngles}(a) and
\ref{fig:projectionAngles}(b). These oblique camera angles for
horizontal surfaces translate into textures that span large areas once
projected, as shown in Figure \ref{fig:projectionAngles}(b). Using the
tile-based texture mapping criteria from Figure
\ref{fig:scoringFunction}, such projections have highly varying scores
depending on the location of a tile on the plane. Thus, the tiling
approach in this section is a good choice for texturing floors and
ceilings, as it uses the parts of image projections that maximize
resolution and accuracy for their respective plane locations,
e.g. areas near point A and not near point B, in Figure
\ref{fig:projectionAngles}(b).


\subsection{Shortest Path Texturing}
\label{sec:shortestPath}
For vertical walls, most images are taken from close distances and
head-on angles, resulting in high resolution fronto-parallel
projections. As a result, for each tile on a wall plane, the scoring
function of Figure \ref{fig:scoringFunction} is relatively flat with
respect to candidate images, as they are all more or less head
on. Thus, the scoring function is less significant for walls, and it
is conceivable to use a different texturing strategy to directly
minimize visible seams when texturing them. This is done by choosing
the smallest possible set of images that (a) covers the entire plane
and (b) minimizes the visibility of borders between them. A
straightforward cost function that accomplishes the latter is the sum
of squared differences (SSD) of pixels in overlapping regions between
all pairs of images. Minimizing this cost function encourages image
boundaries to occur either in featureless areas, such as bare walls,
or in areas where images match extremely well.

The first step in this procedure is to obtain a list of images useful
for texturing the wall. A simple way to do this is to use the images
selected by the tiling process in Section
\ref{sec:tileBasedMapping}. Such a list of images is guaranteed to
cover the entire wall, and consists of desired camera poses overall.

To cover the entirety of a plane, our problem can be defined as
minimally covering a polygon i.e. the planar surface, using other
polygons of arbitrary geometry i.e. image projections, with the added
constraint of minimizing the cost function between chosen images.
This is a complex problem, though we can take a number of steps to
simplify it.

Given that wall-texture candidate images are taken from more or less
head-on angles, and knowing that only minor rotations are made in
Section \ref{sec:2dAlignment}, we can reason that their projections
onto the plane are approximately rectangular. By cropping them all to
be rectangular, as shown in Figure \ref{fig:projectionAngles}(c), our
problem becomes the conceptually simpler one of filling a polygon with
rectangles, such that the sum of all costs between each pair of
rectangles is minimal. We thus also retain the advantages of working
with rectangular units, as explained in Section
\ref{sec:tileBasedMapping}.

The operator's path, and the location and orientation of the cameras
on the acquisition backpack is such that images nearly always contain
the entirety of the floor to ceiling range of wall planes. Images
therefore rarely need to be projected with one above another when
texturing wall planes. In essence, we need only to ensure lateral
coverage of wall planes, e.g. from left to right, as our images
provide full vertical coverage themselves. We can thus construct a
Directed Acyclic Graph (DAG) from the images, with edge costs defined
by the SSD cost function, and solve a simple shortest path problem to
find an optimal subset of images with regard to the SSD cost function
\cite{dijkstra}.

\begin{figure}
  \centering
  \includegraphics[width=3in]{dagCreation.pdf}
  \caption{DAG construction for the image selection process. \\}
  \label{fig:dagCreation}
\end{figure}

Figure \ref{fig:dagCreation} demonstrates the construction of a DAG
from overlapping images of a hallway wall. Images are sorted by
horizontal location left to right, and become nodes in a
graph. Directed edges are placed in the graph from left to right
between overlapping images. The weights of these edges are determined
by the SSD cost function. Next, we add two artificial nodes, one start
node representing the left border of the plane, and one end node
representing the right border of the plane. The left(right) artificial
node has directed edges with equal and arbitrary cost $C_0$ to(from)
all images that meet the left(right) border of the plane. We now solve
the shortest path problem from the start node to the end node. This
results in a set of images completely covering the plane horizontally,
while minimizing the cost of seams between images.

In rare cases where the vertical dimension of the plane is not
entirely covered by one or more chosen images, we are left with holes
where no images are selected to texture. Since these holes are rare,
and generally fairly small, we use a greedy approach, repeatedly
filling the hole with images that result in the lowest SSD costs in a
blending region around the hole, as discussed in Section
\ref{sec:blending}. This method is not as optimal as a true
2D-coverage solution would be, but it is a fast approximation, and
adequately handles the few holes we encounter.

With this completed, we have now mapped every location on the plane to
at least one image, and have minimized the number of images, as well
as the discontinuities at their borders. As seen in Figure
\ref{fig:compareAll}(d), this shortest path method has fewer visible
discontinuities than Figure \ref{fig:compareAll}(c) corresponding to
the tile caching approach\footnote{In Figure \ref{fig:compareAll}(d),
  we arbitrarily chose one image for texturing where images overlap,
  as blending will be discussed in section \ref{sec:blending}.}. This
is especially evident when comparing the posters in the images, as
seen in Figures \ref{fig:compareAll}(e) and
\ref{fig:compareAll}(f). This shortest path approach approach directly
reduces the cost of each image boundary, while the tile caching method
uses a scoring function that only approximates this
effect. Furthermore, this approach guarantees the best selection of
images to minimize seams, while the sequential tile caching method may
select images early on that turn out to be poor choices once
subsequent tiles have been processed. This shortest path approach is
also far less intensive in terms of memory usage and runtime, both
during texture generation and rendering, as it does not require
discretizing planes or images.

When texturing an entire 3D planar model, we apply the shortest path
method on walls, due to its superior output when provided with head-on
images. Floors and ceilings however, given their many images taken at
oblique angles, are textured using the tile caching method.


\subsection{Blending}
\label{sec:blending}
We now apply a blending procedure to both texturing methods. Although
the image alignment steps and image selection in both methods attempt
to minimize all mismatches between images, there are occasional
unavoidable discontinuities in the final texture due to different
lighting conditions or inaccuracies in model geometry. These can
however be treated and smoothed over by applying alpha blending over
image seams.  Whether the units we are blending are
rectangularly-cropped images or rectangular tiles, we can apply the
same blending procedure, as long as we have a guaranteed overlap
between units to blend over.

For the tile caching method, we can ensure overlap by texturing a
larger tile than needed for display. For example, for a rendered tile
$l_1 \times l_1$, we can associate it with a texture $(l_1 + l_2)
\times (l_1 + l_2)$ in size.  We have found $l_2 = \frac{l_1}{2}$ to
provide a balance between blending and keeping features unblurred. For
the shortest path method, we have already ensured overlap between
images. To enforce consistent blending however, we add a minimum
required overlap distance of 40 px while solving the shortest path
problem in Section \ref{sec:shortestPath}. Additionally, if images
overlap in a region greater than the overlap distance, we only apply
blending over an area equal to the overlap distance.

After performing linear alpha blending across overlapping regions, our
texture mapping process is complete. Figures \ref{fig:compareAll}(g)
and \ref{fig:compareAll}(h) show the blended versions of Figures
\ref{fig:compareAll}(c) and \ref{fig:compareAll}(d) respectively. It
is clear that the shortest path approach exhibits much better
alignment and fewer seams than the tile-caching method, as Figure
\ref{fig:compareAll}(h) clearly has the best visual quality among the
textures in Figure \ref{fig:compareAll}.

\section{Results and Conclusions}
\label{sec:resultsAndConclusions}
Examples of ceilings and floors textured with the tile caching
approach, and walls textured with the shortest path approach, are
displayed in Figure \ref{fig:results}. High resolution colored texture
comparisons, as well as further examples and interactive walkthroughs
are available at
\footnote{\url{http://www.eecs.berkeley.edu/~pcheng/indoormapping}}.

As mentioned earlier, our approach is quite efficient. The top wall in
Figure \ref{fig:results}(a) was generated with 7543 $\times$ 776
pixels, and spans a 40-meter long wall. Given 41000 input images, a
2.8GHz consumer-grade laptop takes approximately a minute to pick 36
candidate images, followed by another minute to perform both image
alignment and the shortest path texturing method, though over 75\% of
that time is spent calculating SIFT matches within the SiftGPU
framework. While not real-time, the process is capable of generating
quick updates after changes in various parameters or modifications to
input data, and if integrated directly into a modeling system, could
provide live visualization as data is collected.

In this paper, we have developed an approach to texture mapping models
with noisy camera localization data. We are able to refine image
locations based on geometry references and feature matching, and
robustly handle outliers. Using the tile-based mapping approach, we
can texture both simple rectangular walls as well as complex floor and
ceiling geometry. We also implemented a shortest path texturing method
that produces seamless textures on planes where multiple head-on
images are available. Both of these approaches are highly modular, and
easily tunable for different environments and acquisition systems.

\begin{figure}
  \centering
  \subfloat[][]{\includegraphics[width=3in]{finalfloors.jpg}} ~~~~~~~~
  \centering
  \subfloat[][]{\includegraphics[width=3in]{finalceilings.jpg}}

  \centering \subfloat[][]{
    \includegraphics[height=2in, width=3in]{floorcropped.jpg} ~~~~~~~~
    \includegraphics[width=3in]{pier15floor.jpg}
  }

  \centering{
    \includegraphics[width=1.9in]{fullmodel.png}
    \includegraphics[width=1.9in]{threestoryfull2.png}
    \includegraphics[width=1.9in]{pier15.png} \\
    \includegraphics[width=6in]{threestoryfull.png}
  }

  \centering \subfloat[][]{ }
  \caption{Examples of our final texture mapping output for (a) walls,
    (b) ceilings, (c) floors, (d) full models.}
  \label{fig:results}
\end{figure}


%%%%%%%%%%%%%%%%%%%%%%%%%%%%%%%%%%%%%%%%%%%%%%%%%%%%%%%%%%%%%
%%%%% References %%%%%

\bibliography{report} %>>>> bibliography data in report.bib
\bibliographystyle{spiebib} %>>>> makes bibtex use spiebib.bst

\end{document} 
