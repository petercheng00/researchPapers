\message{ !name(masters.tex)}\documentclass[]{spie}  %>>> use for US letter paper

\usepackage{graphicx}
\usepackage{subfig}
\usepackage{amsmath}
\usepackage{amssymb}
\usepackage{hyperref}
\usepackage{float}
\usepackage{multirow}

\title{Texture Mapping 3D Models of Indoor Environments with Noisy Camera Poses} 

\author{Peter Cheng
\skiplinehalf
University of California, Berkeley\\
}

\begin{document}

\message{ !name(masters.tex) !offset(135) }
\subsection{Geometry Partitioning}
\label{sec:geometryPartioning}
In order to effectively texture a 3D model, we first divide the model
into a set of regions, each to be textured independently. The purpose
of this is to ensure texture continuity along large uniform areas,
while allowing texture boundaries to fall along natural environmental
boundaries. Because textures may not match at such boundaries where
two regions meet, it is important to minimize the visibility of
boundaries. This is done by encouraging region boundaries to occur at
large sharp corners, where texture continuity is less important.

When working with low-resolution models, where surfaces generally
correspond to large planar features such as walls, ceilings, or
floors, and often meet at right angles, each planar surface in the
model is directly treated as a distinct region and textured
separately.

When working with a high-resolution model however, geometry often
represents large features as well as small details found in furniture,
plants, and other miscellaneous objects. For instance, in Figure
\ref{fig:deskGeom}, the geometry modeling the desk and binders
consists of many small triangles at various orientations. Texturing
each of these triangles independently makes little sense, as the
textures applied to each should match up well with the textures of its
neighbors in order to avoid any breaks in the overall image of the
desk and binders. On the other hand, the textures for the triangles
belonging to the wall behind need not match up with the textures on
the desk and binders, and so each group of triangles should be
designated to separate regions. This region grouping is performed by designating all contiguous coplanar groups of triangles as different regions. Any region that is less than $1 m^2$ in size is joined to its largest neighboring region, such that the angle between them is under $90^{\circ}$.

Because a texture is 2D, and the region to be textured may not be
flat, a mapping needs to be defined. We first fit a plane to each
region for the purposes of texturing. This can be accomplished via a
least squares plane-fitting through each triangle vertex, though in
practice we simply use the plane equation of the largest contiguous
section of coplanar triangles, as most regions are characterized by
small outcroppings from a largely planar surface, such as objects on a
table or features protruding from a wall. The 3d vertices within a
region are then projected onto its calculated plane, and the
rectangular bounding box for these projected vertices is
calculated. This bounding box corresponds to the 2D image that will be
created to texture the region.

Of course, the area encompassed by the bounding box is larger than the
corresponding 2d area of the projected geometry, which itself is not
as accurate as the original 3d geometry. Thus, we also create a k-d
tree consisting of the triangle primitives from the original
mesh. Though we generate texture for the 2d bounding box surface, we
will reference the original 3D geometry to ensure accuracy when
texturing.

\message{ !name(masters.tex) !offset(944) }

\end{document} 
