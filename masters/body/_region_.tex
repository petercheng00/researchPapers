\message{ !name(masters.tex)}\documentclass[]{spie}  %>>> use for US letter paper

\usepackage{graphicx}
\usepackage{subfig}
\usepackage{amsmath}
\usepackage{amssymb}
\usepackage{hyperref}
\usepackage{float}
\usepackage{multirow}

\title{Texture Mapping 3D Models of Indoor Environments with Noisy Camera Poses} 

\author{Peter Cheng
\skiplinehalf
University of California, Berkeley\\
}

\begin{document}

\message{ !name(masters.tex) !offset(1175) }
\subsection{Runtime}
As mentioned earlier, our approach is quite efficient. The top wall in
Figure \ref{fig:indivPlanes}(a) was generated with 7543 $\times$ 776
pixels, and spans a 40-meter long wall. Given 41000 input images in
the entire dataset, a 2.8 GHz dual-core consumer-grade laptop takes
under a second to choose 36 candidate images, followed by a minute to
perform image projections, image alignment, and the shortest path
texturing method. The tile-caching approach takes roughly comparable
time, and over 75\% of the time for either method is spent on
calculating image projections, which are cached over multiple runs,
and also could be performed as a preprocessing step instead. While not
real-time, the process is capable of generating fast updates after
changes in various parameters or modifications to input data, and if
integrated directly into a 3D modeling system, could provide quick
visual feedback as data is collected.


\message{ !name(masters.tex) !offset(1325) }

\end{document} 
